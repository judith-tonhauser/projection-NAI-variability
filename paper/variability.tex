
\documentclass[11pt,fleqn]{article}
\usepackage[margin=1in,top=1in,bottom=1in]{geometry}
\usepackage{mathtools}
\usepackage{longtable}
\usepackage{enumitem}
\usepackage{hyperref}
\usepackage[dvips]{graphics}
\usepackage[table]{xcolor}
\usepackage{amssymb}
\usepackage{subfig}
\usepackage{booktabs}

\usepackage[normalem]{ulem}

\usepackage{multicol}
\usepackage{txfonts}
%\usepackage{amsfonts}
\usepackage{natbib}
\usepackage{gb4e}
%\usepackage{/Users/judith/Library/Latex/drs}
%\usepackage{/Users/judith/Library/Latex/avm}
\usepackage[all]{xy}
\usepackage{rotating}
\usepackage{tipa}
\usepackage{multirow}
\usepackage{authblk}


\newcommand{\foc}{$_{\mbox{\small F}}$}
\newcommand{\lp}{<_{\hspace*{-.1cm}p}}
\newcommand{\lnai}{<_{\hspace*{-.1cm}nai}}

\setlength{\parindent}{.8cm}
\setlength{\parskip}{0ex}
\setlength{\headsep}{0in}

\setlength{\bibsep}{0mm}
\bibpunct[:]{(}{)}{;}{a}{,}{,}

\newcommand{\yi}{\'{\symbol{16}}}
\newcommand{\nasi}{\~{\symbol{16}}}
\newcommand{\hina}{h\nasi na}
\newcommand{\ina}{\nasi na}
%\renewcommand{\abut}{$\supset$\hspace*{-0.07cm}$\subset$}
\newcommand{\tto}{t$_{top}$}
\newcommand{\wtop}{w$_{top}$}
\newcommand{\tc}{t$_c$}
\newcommand{\schwa}{\begin{sideways}e\end{sideways}}

% Semantic brackets
%\newcommand{\iss}[1]{\mbox{\protect\tiny \mbox{#1}}}
%\newcommand{\sem}[2]{\6#1\9$_\iss{#2}$} David's original
\newcommand{\6}{\mbox{$[\hspace*{-.6mm}[$}} 
\newcommand{\9}{\mbox{$]\hspace*{-.6mm}]$}}
\newcommand{\sem}[2]{\6#1\9$^{#2}$}

\newcommand{\semt}[2]{$\left[\hspace*{-.6mm}\left[\begin{tabular}[c]{@{}l@{}}#1\vspace*{-.5em}\end{tabular}\right]\hspace*{-.6mm}\right]\hspace*{-.6mm}^{#2}$}

%\renewcommand{\baselinestretch}{1.2}

\def\bad{{\leavevmode\llap{*}}}
\def\marginal{{\leavevmode\llap{?}}}
\def\verymarginal{{\leavevmode\llap{??}}}
\def\infelic{{\leavevmode\llap{\#}}}

\definecolor{Lighter}{gray}{.92}
\definecolor{Blue}{RGB}{0,0,255}
\definecolor{Green}{RGB}{10,200,100}
\definecolor{Red}{RGB}{255,0,0}
\newcommand{\jd}[1]{\textcolor{Blue}{[jd: #1]}}  

\newcommand{\citepos}[1]{\citeauthor{#1}'s \citeyear{#1}}
\newcommand{\citeposs}[1]{\citeauthor{#1}'s}
\newcommand{\citetpos}[1]{\citeauthor{#1}'s (\citeyear{#1})}

\newcommand{\eref}[1]{(\ref{#1})}
\newcommand{\tableref}[1]{Table \ref{#1}}
\newcommand{\figref}[1]{Fig.~\ref{#1}}
\newcommand{\appref}[1]{Appendix \ref{#1}}
\newcommand{\sectionref}[1]{Section \ref{#1}}


\title{The information-structural source of projection variability}

\author[$\bullet$]{Judith Tonhauser}
\author[$\circ$]{David Beaver}
\author[$\triangleright$]{Judith Degen}


\affil[$\bullet$]{The Ohio State University}
\affil[$\circ$]{University of Texas at Austin}
\affil[$\triangleright$]{Stanford University}


\renewcommand\Authands{ and }

\newcommand{\jt}[1]{\textbf{\color{blue}JT: #1}}

\begin{document}

\maketitle

\begin{abstract}

Projective content has long been observed to vary in how robustly it projects, i.e., in how robustly speakers are taken to be committed to it when the expression that contributes the content occurs embedded under an entailment-canceling operator (e.g., \citealt{karttunen71b,ccmg90,simons01,abusch10}). Projection variability presents a challenge for approaches to projection that assume that the projectivity of projective content derives from a conventionally specified requirement of such content to be entailed by or satisfied in the common ground of the interlocutors (e.g., \citealt{heim83,vds92}). An alternative approach, advanced in \citet{brst-salt10}, maintains that projective content projects if and only if it is not at-issue with respect to the Question Under Discussion (e.g., \citealt{roberts12}) addressed by the utterance that contributes the projective content (see also \citealt{brst-ar}). Under this approach, projection variability can be attributed to variable at-issueness. This paper reports the findings of two pairs of experiments designed to explore the extent to which projective content contributed by a wide range of English expressions varies in its projectivity and the hypothesis that the information-structural status of projective content influences its projectivity. The findings of the two pairs of experiments further substantiate the empirical challenges for conventionalist approaches to projection and provide empirical support the approach to projection advanced in \citealt{brst-salt10} and \citealt{brst-ar}.

\end{abstract}

%\tableofcontents

%\newpage
			
\section{Introduction}\label{s1}

Projective content is utterance content that the speaker may be taken to be committed to even when the expression that contributes the content occurs in the syntactic scope of an entailment-canceling operator (see, e.g., \citealt{ccmg90}). The speaker of (\ref{eng1}), for instance, is taken to be committed to the content of the complement of {\em discover}, that Mike visited Alcatraz, since this content is entailed by the utterance. The so-called Family-of-Sentences variants of (\ref{eng1}) given in (\ref{eng2}a-d) do not entail this content because {\em discover} is embedded under entailment-canceling operators: negation in (\ref{eng2}a), the polar question operator in (\ref{eng2}b), the epistemic possibility modal {\em perhaps} in (\ref{eng2}c) and the antecedent of a conditional in (\ref{eng2}d). Since speakers who utter the sentences in (\ref{eng2}a-d) may nevertheless be taken to be committed to the content of the complement, this content, by virtue of being able to `project' over the entailment-canceling operators, is considered projective content. 

\begin{exe}
\ex\label{eng1}  Felipe discovered that Mike visited Alcatraz.

\ex\label{eng2}
\begin{xlist} 
\ex Felipe didn't discover that Mike visited Alcatraz.
\ex Did Felipe discover that Mike visited Alcatraz?
\ex Perhaps Felipe discovered that Mike visited Alcatraz.
\ex If Felipe discovered that Mike visited Alcatraz, he'll get mad.
\end{xlist}
\end{exe}

Why does projective content project? According to mainstream approaches, projective content projects because it is conventionally specified to do so, for instance, by being required to be entailed by or satisfied in the common ground of the interlocutors (e.g., \citealt{heim83,vds92,geurts99}). On such `conventionalist' approaches, the lexical entry of {\em discover} specifies that the content of its clausal complement is required to be entailed by or satisfied in the common ground of the interlocutors, thereby ensuring that the speaker is taken to be committed to the content. Since conventionalist approaches only distinguish projective and non-projective content, such approaches are challenged by the long-standing observation that some projective content projects less robustly than other such content. In the early 1970s already, \citet{karttunen71b} pointed out that the content of the complement of {\em regret} in (\ref{semi-factive}a) is more robustly projective than the content of the complement of {\em discover} in (\ref{semi-factive}b); following \citealt{karttunen71b}, predicates like {\em discover} have since been referred to as `semi-factive', in contrast to their `factive' counterparts like {\em regret} (and `non-factive' predicates like {\em believe}). In \citealt{schlenker10}, the predicate {\em announce} was referred to as a `part-time trigger' because the content of its complement may, but often does not, project.

\begin{exe}
\ex\label{semi-factive}
\begin{xlist}
\ex John didn't discover that he had not told the truth.  
\ex John didn't regret that he had not told the truth.
\hfill (\citealt[63]{karttunen71b})

\end{xlist}
\end{exe}

Similarly, \citet[432]{simons01} noted, partially based on examples from \citealt{ccmg90} and \citealt{geurts94}, that the projection of ``some -- but crucially, not all'' projective content to the common ground of the interlocutors may be suppressed in explicit ignorance contexts. Example (\ref{hardsoft}a), for instance, shows that the projective content of {\em win} in the antecedent of the conditional, that John participated in the race, need not be part of the common ground of the interlocutors, i.e., need not project. On the other hand, the existential implication of the cleft in (\ref{hardsoft}b), that there is an individual who read the letter, must be part of the common ground of the interlocutors, i.e., must project. Expressions like {\em win} are referred to as `soft triggers', in contrast to `hard triggers', like the cleft (see also, e.g., \citealt{abusch10}).

\begin{exe}
\ex\label{hardsoft}
\begin{xlist}

\ex I have no idea whether John ended up participating in the Road Race yesterday. But if he won it, then he has more victories than anyone else in history. \hfill (\citealt[39]{abusch10})

\ex\infelic I have no idea whether anyone read that letter. But if it is John
who read it, let's ask him to be discreet about the content. \hfill (\citealt[40]{abusch10})

\end{xlist}
\end{exe}

Experimental research has also provided preliminary evidence for projection variability. \citet{xue-onea11} observed that the content of the complement of German {\em wissen} `know' is less projective than the content of the complement of {\em erfahren} `find out', both of which are less projective than the relevant projective contents of sentences with {\em auch} `too' (that a parallel event is contextually salient) and {\em wieder} `again' (that the relevant event has happened before). Similarly, \citet{smith-hall11} found that the projective contents of {\em win} and {\em know} are less projective than the content implication of English definite noun phrases (e.g., for {\em the queen}, that the referent is a queen). Interestingly, they also found that the existential implication of cleft sentences, considered a hard trigger, was (numerically) less projective than the relevant contents of the soft triggers {\em win} and {\em know}. Thus, the (sparse) experimental evidence confirms some but not all of the intuitions about projection variability reported in the literature.\footnote{\citealt{tiemann-etal11} observed that presuppositions differed in how acceptable they were judged to be in contexts that did not entail the relevant content.}

Observations about projection variability challenge conventionalist approaches to projection because such approaches do not offer an explanation for why some projective content seems to project less robustly than other such content. After all, under such approaches, any expression that contributes projective content lexically specifies that the relevant content is required to be entailed by or satisfied in the common ground of the interlocutors. The lexical specifications of expressions like {\em regret, discover, win}, clefts and definite noun phrases thereby predict that their relevant contents can project, but do not predict differences in how robustly the contents project. Referring to triggers of projective content as `semi-factive' or `soft', as opposed to `factive' or `hard', serves to remind of projection variability but does not address the challenge that this variability poses for conventionalist approaches to projection.

One possible explanation for projection variability is that projectivity derives from a property that projective content shares, but shares to varying degrees. \citet{brst-salt10} proposed that this property is `at-issueness', i.e., the ability of content to address the Question Under Discussion (QUD; e.g., \citealt{roberts12}); see also \citealt{brst-ar} for this approach. Specifically, Simons and her colleagues proposed that utterance content projects if and only if it is not at-issue with respect to the QUD addressed by the utterance. Although \citet{brst-salt10} did not consider projection variability, their proposal predicts such variability. For instance, their proposal predicts that the content of non-restrictive relative clauses (NRRCs) projects more robustly, by virtue of being not at-issue (\citealt{potts05}), than the content of the complement of {\em discover}, which can be at-issue and not-at-issue (\citealt{simons07}). Consider the question-answer pairs in (\ref{nrrc}): the example in (\ref{nrrc}a) shows that the content of the NRRC in B's utterance cannot be used to address A's question, and hence is not at-issue; the example in (\ref{nrrc}b) shows that the main clause content of B's utterance can address A's question and hence is at-issue. The content of the complement of {\em discover}, on the other hand, can be not-at-issue, as shown in (\ref{discover}a), but it can also be at-issue, as shown in (\ref{discover}b).\footnote{\citet{syrett-koev2015} show that the content of NRRCs can be the target of direct denial, which may be taken to suggest that this content can be at-issue. We return to this matter in section \ref{s5}.} Thus, if the content of NRRCs is more robustly not-at-issue than the content of the complement of {\em discover}, the former is predicted by \citetpos{brst-salt10} proposal to project more robustly than the latter. 

\begin{exe}
\ex\label{nrrc}
\begin{xlist}
\ex
\begin{xlist}
\exi{A:} Did Mike visit Alcatraz?
\exi{B:} \infelic Mike, who visited Alcatraz, is interested in the history of prisons.
\end{xlist}


\ex
\begin{xlist}
\exi{A:} What is Mike interested in?
\exi{B:} Mike, who visited Alcatraz, is interested in the history of prisons.
\end{xlist}


\end{xlist}

\ex\label{discover}
\begin{xlist}

\ex
\begin{xlist}
\exi{A:} Why is Henry in such a bad mood?
\exi{B:} He discovered that Harriet had a job interview at Princeton. 
\end{xlist}

\ex
\begin{xlist}
\exi{A:} Where was Harriet yesterday?
\exi{B:} Henry discovered that she had a job interview at Princeton. \hfill (\citealt[1035]{simons07})
\end{xlist}
\end{xlist}
\end{exe}

This paper has two goals, which are explored on the basis of two pairs of experiments. The first goal (Experiments 1a and 1b) is to explore projection variability for a broad range of projective content, in order to better understand the extent to which projection variability presents a challenge for conventionalist approaches to projection. The second goal (addressed through Experiments 1a and 1b, as well as Experiments 2a and 2b) is to test the proposal put forth in \citealt{brst-salt10}, that the projectivity of utterance content is influenced by at-issueness, i.e., the information-structural status of such content. These two goals jointly serve to assess the empirical adequacy of conventionalist approaches to projection, on the one hand, and the approach advanced in  \citealt{brst-salt10} and \citealt{brst-ar}, on the other.

In pursuing our first goal, we expand and improve on previous experimental research on projection variability. In \citealt{xue-onea11} and \citealt{smith-hall11}, projection variability was explored for 4 German and 6 English expressions that contribute projective content, respectively. Experiments 1a and 1b significantly broaden our understanding of projection variability by considering the projective content of 19 expressions (which are introduced in section \ref{s2}). Our experiments also take into consideration that the prior plausibility of content may influence projection: a speaker might, for instance, be more likely to be taken to be committed to the content that Alexander flew to New York than to the content that Alexander flew to the moon, simply because people are more likely to fly to New York than the moon. In other words, the content of the projective content of an expression\footnote{\label{f-content}In this paper, we use `content' to refer to the content of an expression and `projective content' to refer to an abstract characterization of the projective content of an expression. For instance, in B's utterance in (\ref{discover}a), the relevant expression is {\em discover}, the projective content is the content of its clausal complement, and the content (of the projective content) is that Harriet had a job interview at Princeton.\jd{maybe we want to change `content' to something that is less confusable with `projective content' to make it easier on the reader? }} may matter for how robustly the projective content is taken to be a commitment of the speaker and how likely the projective content is to address the QUD. The projective content of the 6 expressions explored in \citealt{smith-hall11} was only instantiated by one content each and a distinct content instantiated each projective content; similarly, the projective content of the 4 expressions explored in \citealt{xue-onea11} was instantiated by three distinct contents each. Our experiments, in contrast, included a total of 37 contents and the projective content of each expression was instantiated by up to 20 contents. Furthermore, to facilitate comparison across the different projective contents (and the expressions that contribute them), the projective contents of distinct expressions were instantiated with the same contents: overall, each of the 37 contents instantiated up to 12 projective contents.

Our second goal -- to explore the hypothesis that the information-structural status of projective content influences its projectivity -- also builds on previous experimental work. 
 Using a direct dissent diagnostic for at-issueness, \citet{amaral-etal11} found that speakers of British English judged direct dissent with the projective content of {\em only} (the prejacent) to be more acceptable than direct dissent with the projective content of {\em continue} and {\em stop} (the pre- and post-state implications, respectively). These findings suggest that the prejacent of {\em only} is more at-issue than the post- and pre-state implications of {\em continue} and {\em stop}. (See also \citealt{cummins-etal2012} and, on Spanish, \citealt{amaral-cummins2015}.) \citet{xue-onea11} found that speakers of German were more likely to directly dissent with the content of the complement of {\em wissen} `know' than with the content of the complement of {\em erfahren} `find out' and, in turn, more likely to directly dissent with these contents than with projective contents of {\em auch} `too' and {\em wieder} again'. These results suggest that the content of the complement of {\em wissen} `know' is more at-issue than the content of the complement of {\em erfahren} `find out', which in turn are comparatively more at-issue than the relevant contents of {\em auch} `too' and {\em wieder} again'. Interestingly, comparing the relative projectivity and not-at-issueness 
of the contents across their two experiments, \citet[180]{xue-onea11} point to ``a clear correlation between projection and not-at-issueness'', in line with the proposal made in \citealt{brst-salt10} that we explore further here. Our Experiments 1a and 1b improve on \citetpos{xue-onea11} study by exploring the projectivity and the not-at-issueness of projective content as within-item and within-participant factors. The design of Experiments 1a and 1b allows us to quantify the influence of not-at-issueness on the projectivity of utterance content but also to assess item- and participant-variability. Since different diagnostics have been used in the literature to diagnose (not-)at-issueness (for discussion, see, e.g., \citealt{tonhauser-sula6} and section \ref{s3}), the second pair of experiments, Experiments 2a and 2b, explore the not-at-issueness of the contents of the first pair of experiments, Experiments 1a and 1b, using a different diagnostic for at-issueness. The results of the two pairs of experiments thus also allow for a comparison of two diagnostics for at-issueness.

The paper proceeds as follows. In section \ref{s2}, we characterize the projective contents explored in this paper. Section \ref{s3} addresses the two goals of the paper on the basis of Experiments 1a and 1b, and section \ref{s4} extends our investigation of the second goal on the basis of Experiments 2a and 2b. In section \ref{s5}, we discuss the implications of our findings for conventionalist and \citetpos{brst-salt10} approaches to projection and projection variability. Section \ref{s6} concludes the paper.

\section{The projective contents explored in the experiments}\label{s2}

This section introduces properties of the projective contents explored in the two pairs of experiments. These projective contents are contributed by 19 expressions, henceforth referred to as `triggers', but without making any commitment to these expressions conventionally specifying the projectivity of the relevant content. 

The 9 projective content triggers included in Experiment 1a are syntactically heterogeneous, as shown in (\ref{pairs1a2a}), whereas the 12 triggers included in Experiment 1b are all attitude predicates, as shown in (\ref{pairs1b2b}). Since some expressions can contribute multiple projective contents (\citealt{brst-lang11}), we specify, for each trigger, the projective content investigated in our experiments: e.g., in (\ref{pairs1a2a}a), `sentence-medial NRRCs' identifies the trigger and `content of the NRRC' identifies the projective content. Experiments 2a and 2b explored not-at-issueness for the same trigger/projective content pairs as Experiments 1a and 1b, respectively. Since the projective content of the attitude predicates {\em annoyed} and {\em discover} was explored in both pairs of experiments, a total of 19 trigger/projective content pairs were explored. 

\begin{exe}
\ex\label{pairs1a2a} {\bf Trigger/projective content pairs in Experiments 1a and 2a}

\begin{enumerate}[itemsep=-.5mm]

\item Sentence-medial NRRCs / content of the NRRC
\\ e.g., {\em Martha, who has a new BMW, drives fast}; content: `Martha has a new BMW'

\item Sentence-medial nominal appositives / appositive content
\\ e.g., {\em Martha's new car, a BMW, was expensive}; content: `Martha has a BMW'

\item Posessive noun phrases / possession implication
\\ e.g., {\em Martha's new car is a BMW}; possession implication: `Martha has a new car'

\item {\em annoyed} / content of the clausal complement
\\ e.g., {\em Martha's neighbor was annoyed that Martha has a new BMW}; content: `Martha has a new BMW'

\item {\em discover} / content of the clausal complement
\\ e.g., {\em Mary discovered that her daughter has been biting her nails}; content: `Mary's daughter has been biting her nails'

\item {\em know} / content of the clausal complement
\\ e.g., {\em Billy knows that Martha has a new BMW}; content: `Martha has a new BMW'

\item {\em only} / prejacent
\\ e.g., {\em Martha only likes blueberries}; prejacent: `Martha likes blueberries'

\item {\em stop} / pre-state implication
\\ e.g., {\em Martha stopped smoking}; pre-state implication: `Martha smoked in the past'

\item {\em stupid} / prejacent
\\ e.g., {\em Martha was stupid to buy that TV}; prejacent: Martha bought that TV

\end{enumerate}


\ex\label{pairs1b2b} {\bf Trigger/projective content pairs in Experiments 1b and 2b}

\begin{enumerate}[itemsep=-.5mm]

\item {\em be amused} / content of the clausal complement

\item {\em be annoyed} / content of the clausal complement

\item {\em know} / content of the clausal complement

\item {\em be aware} / content of the clausal complement

\item {\em see} / content of the clausal complement

\item {\em discover} / content of the clausal complement

\item {\em find out} / content of the clausal complement

\item {\em realize} / content of the clausal complement

\item {\em learn} / content of the clausal complement

\item {\em establish} / content of the clausal complement

\item {\em confess} / content of the clausal complement

\item {\em reveal} / content of the clausal complement

\end{enumerate}

\end{exe}

In what follows, we characterize the properties of the 19 trigger/projective content pairs explored in the experiments.

\paragraph{Projectivity} The 19 projective contents share the property of being projective, i.e., being able to be taken to be a commitment of the speaker even when the trigger is embedded under an entailment-canceling operator. The 9 expressions included in Experiments 1a and 2a differ in how robustly their contents have been reported to project: 
the content of NRRCs, nominal appositives and of the emotive `factive' predicate {\em annoyed} are typically taken to project more robustly than, e.g., the prejacent of {\em only}, the pre-state implication of {\em stop} and the complement of the cognitive `semi-factive' predicate {\em discover} (e.g., \citealt{karttunen71b,simons01,potts05,abusch10,beaver-belly}). The 12 triggers included in Experiments 1b and 2b include both `factive' and `semi-factive' predicates, and were also chosen to denote different types of attitudes: emotive `factive' predicates ({\em be amused, be annoyed}), cognitive `factive' predicates ({\em know, be aware}), sensory `factives' predicates ({\em see}), cognitive `semi-factives' predicates ({\em discover, find out, realize, learn, establish}) and communication `semi-factives' ({\em confess, reveal}). The predicates {\em be annoyed} and {\em discover} were included in both Experiments 1a and 1b (and 2a and 2b) to be able to directly compare the results of the experiments. \jt{Do all of the attitude predicates entail the content of the complement? Need  experiment!}


\paragraph{Strong Contextual Felicity} A property shared by the 19 trigger/projective content pairs is that they do not impose a Strong Contextual Felicity constraint on the utterance context (\citealt{brst-lang11}). What this means is that utterances with the triggers included in the experiments are judged to be acceptable in contexts in which the projective content is not already part of the common ground of the interlocutors when the trigger is uttered. For instance, B's utterance in (\ref{nrrc}b), repeated below, is acceptable even if A did not previously know the content of the NRRC, that Mike visited Alcatraz. A trigger/projective content pair that is associated with a Strong Contextual Felicity contraint is the pronoun {\em they} and the projective content that there is a uniquely salient plurality of individuals (to which the pronoun refers). Use of {\em they} in (\ref{scf}) is judged to be unacceptable because the projective content of the pronoun is not part of the common ground of the interlocutors, i.e., the utterance context does not entail the existence of a uniquely salient plurality of individuals to which the pronoun could refer. 
%Other projective content triggers that impose a Strong Contextual Felicity constraint include {\em too} (there is a salient alternative proposition) and the deictic implication of demonstratives (see \citealt{brst-lang11}).

\begin{exe}
\exi{(\ref{nrrc}b)}
\begin{xlist}
\exi{A:} What is Mike interested in?
\exi{B:} Mike, who visited Alcatraz, is interested in the history of prisons.
\end{xlist}

\ex\label{scf} At a bus stop, one woman asks another one, with no other people around: \\ \infelic Did they visit Alcatraz?
\end{exe}

Including in our experiments only trigger/projective content pairs not associated with a Strong Contextual Felicity constraint was motivated by our goal of exploring the relative projectivity and not-at-issueness of the triggers. It is well-known that the context in which a projective content trigger occurs influences whether the projective content projects (see, e.g., the examples in (\ref{hardsoft}), but also, e.g., \citealt{simons01,beaver-belly}). In our experiments, all of the projective content triggers were therefore presented in the same contexts, namely ones that clarified the situation in which the trigger was uttered but that minimized the extent to which the context might influence the projectivity or not-at-issueness of the relevant content. Including trigger/projective content pairs associated with a Strong Contextual Felicity constraint would have forced us to present the triggers in different contexts, thereby hampering cross-trigger comparison.\footnote{As discussed in section \ref{s3}, the minimal contexts in which the stimuli were presented also have the property of not plausibly licensing `local accommodation' (\citealt{heim83,vds92}), the process that allows conventionalist approaches to projection to account for projective content not projecting.} 

\paragraph{Obligatory Local Effect} The 19 trigger/projective content pairs differ in whether they have Obligatory Local Effect (\citealt{brst-lang11}), the property that distinguishes trigger/projective content pairs based on whether the projective content is obligatorily contributed to an attitude holder's belief state when the trigger occurs in the complement of a belief-predicate. The examples in (\ref{ole}) illustrate that the content of NRRCs does not have Obligatory Local Effect, in contrast to the content of the complement of {\em discover}, which does. In (\ref{ole}a), where the NRRC occurs in the complement clause of {\em believe}, the attitude holder (Sarah) need not be committed to the content of the NRRC, that Mike visited Alcatraz, as shown by the acceptability of the continuation. In contrast, in (\ref{ole}b), where {\em discover} occurs in the complement of the belief-predicate, Sarah must be committed to the same content (that Mike visited Alcatraz), as shown by the unacceptability of the same continuation. 

\begin{exe}
\ex\label{ole}
\begin{xlist}
\ex Sarah believes that Mike, who visited Alcatraz, is interested in the history of prisons, \ldots but she doesn't know that Mike has already visited Alcatraz.

\ex Sarah believes that Felipe discovered that Mike visited Alcatraz, \ldots \#but she doesn't know that Mike has already visited Alcatraz. 

\end{xlist}
\end{exe}
In addition to the projective content of NRRCs, the projective contents of appositives and of possessive noun phrases do not have Obligatory Local Effect. The remaining 16 trigger/projective content pairs, including that of {\em discover}, have Obligatory Local Effect. 

In sum, the 9 projective content triggers explored in Experiments 1a and 2a are syntactically heterogeneous, have been reported to differ in how robustly the projective content projects and include 3 that have Obligatory Local Effect and 6 that don't. The 12 projective content triggers explored in Experiments 1b and 2b are all attitude predicates, i.e., syntactically comparatively homogenous, and all have Obligatory Local Effect, but also differ in how robustly their projective contents have been reported to project.


\section{Experiment 1}
\label{s3}

Experiments 1a and 1b were designed to empirically explore the projectivity of the 19 projective contents. In particular, the goals of Experiments 1a and 1b were a) to establish whether and how much variability in projectivity exists; and b) to test \citetpos{brst-salt10} proposal that the projectivity of utterance content is influenced by at-issueness, i.e., the information-structural status of such content. 

To examine the influence of at-issueness on projectivity, we collected judgments about both the projectivity and the at-issueness of the relevant projective content for each trigger/content pair. Consider, for instance, Patrick's polar question with the nominal appositive in (\ref{stim}). The relevant projective content here is that Martha's new car is a BMW. The `certain that' response question in (\ref{stim}a) assesses the extent to which Patrick is taken to be committed to the projective content, i.e., the extent to which the content projects from Patrick's polar question. The `asking whether' response question in (\ref{stim}b) assesses the extent to which Patrick is asking about the projective content, i.e., the extent to which the projective content is at-issue in Patrick's polar question. (For other uses of this diagnostic of at-issueness see, e.g., \citealt{amaral-etal07} and \citealt{tonhauser-sula6}.)

\begin{exe}

\ex\label{stim} Patrick asks: {\em Was Martha's new car, a BMW, expensive?} 

\begin{xlist}
\ex `certain that' question (projectivity): Is Patrick certain that Martha's new car is a BMW?

\ex `asking whether' question (at-issueness): Is Patrick asking whether Martha's new car is a BMW?

\end{xlist}

\end{exe}
In Experiments 1a and 1b, each participant was asked to respond to both the `certain that'  and the `asking whether' response questions for each experimental item they were presented with.

\subsection{Experiment 1a}\label{s-exp1a}

Experiment 1a was designed to explore the projectivity and at-issueness of the 9 projective contents occurring with the syntactically heterogeneous triggers in (\ref{pairs1a2a}), i.e., the content of NRRCs and nominal appositives, the possession implication of possessive noun phrases, the prejacent of {\em only} and {\em stupid}, the pre-state implication of {\em stop} and the contents of the complements of {\em annoyed, discover} and {\em know}.

\subsubsection{Methods}\label{s-methods-1a}

\paragraph{Participants.} 250 participants with U.S.\ IP addresses and at least 97\% of previous HITs approved were recruited on Amazon's Mechanical Turk platform (ages: 18-74; median: 32). They were paid \$1 for participating in the experiment. 

\paragraph{Materials.} The 9 projective contents explored in this experiment were instantiated by 17 contents, which are shown in (\ref{contents}) together with the label used to refer to the content. Recall that we use the term `projective content' to refer to the abstract characterization of the projective content of an expression (e.g., for {\em discover}, the content of the clausal complement) and the term `content' to refer to the content with which the projective content is instantiated (see footnote \ref{f-content}).


\begin{exe}
\ex\label{contents} 17 contents used in Experiment 1a and 2a

\begin{enumerate}[itemsep=-.5mm]

\ex muffins: these muffins have blueberries in them

\ex pizza: this pizza has mushrooms on it

\ex play: Jack was playing outside with the kids

\ex vegetarian: Don is a vegetarian

\ex cheat: Raul cheated on his wife

\ex nails: Mary's daughter has been biting her nails

\ex ballet: Ann used to dance ballet

\ex kids: John's kids were in the garage

\ex hat: Samantha has a new hat

\ex bmw: Martha has a new BMW

\ex boyfriend: Betsy has a boyfriend

\ex alcatraz: Mike visited Alcatraz

\ex aunt: Janet has a sick aunt

\ex cupcakes: Marissa brought the cupcakes

\ex soccer: the soccer ball has a hole in it

\ex olives: this bread has olives in it

\ex stuntman: Richie is a stuntman

\end{enumerate}
\end{exe}

Each of the 9 projective contents was instantiated by 3 to 8 of these 17 contents, as shown in \tableref{t-trigger-content-pairs}. As discussed in \sectionref{s1}, different projective contents were instantiated by the same contents (e.g., the content `muffins' instantiates the projective content of non-restrictive relative clauses and {\em only}). This allows us to test for independent contributions of trigger and content to potential variability in projectivity. As shown in \tableref{t-trigger-content-pairs}, each content instantiated between 2 and 4 projective contents.

\begin{table}[!h]
\begin{center}
\begin{tabular}{l|ccccccccc}
 & \multicolumn{9}{c}{\bf Projective content triggers} \\ 
 
{\bf content} & NRRC & NomApp & possNP & {\em discover} & {\em know} & {\em annoyed} & {\em stop} & {\em only} & {\em stupid} \\\hline \hline

muffins & $\checkmark$ & & & & & & & $\checkmark$ &  \\

\hline

kids & & & & & & & & $\checkmark$ & $\checkmark$ \\

\hline

pizza & & & & $\checkmark$ & & $\checkmark$ & & $\checkmark$ &  \\

\hline

play & & & & $\checkmark$ & $\checkmark$ & & $\checkmark$ & &  \\

\hline

nails & & & & $\checkmark$ & & & $\checkmark$ & & $\checkmark$  \\

\hline

ballet & & $\checkmark$& & & & & $\checkmark$ & &  \\

\hline

cheat & & & & & $\checkmark$ & & & & $\checkmark$ \\

\hline

stuntman & & $\checkmark$ & & & & & & & $\checkmark$ \\

\hline

bmw & & $\checkmark$ & $\checkmark$ & & $\checkmark$ & $\checkmark$ & & &  \\

\hline

vegetarian & $\checkmark$ & $\checkmark$& & & & & & &  \\

\hline

hat & & & $\checkmark$ & & $\checkmark$ & $\checkmark$ & & &  \\

\hline

boyfriend & $\checkmark$ & & $\checkmark$ & & & & & &  \\

\hline

aunt & $\checkmark$ & & $\checkmark$ & & $\checkmark$ & & & &  \\

\hline

alcatraz & $\checkmark$ & & & $\checkmark$ & $\checkmark$ & & & &  \\

\hline

soccer & $\checkmark$ & & & $\checkmark$ & $\checkmark$ & $\checkmark$ & & &  \\

\hline

olives & & & & & & $\checkmark$ & & &  \\

\hline

cupcakes & $\checkmark$ & & & & $\checkmark$ & & & &  \\

\hline

%only: muffins","kids","pizza"], 3
%"stop":["play","nails","ballet"],	3
%"stupid":["kids","cheat","nails","stuntman"], 4
%"NomApp":["bmw","veggie","ballet","stuntman"], 4
%"possNP":["hat","bmw","boyfriend","aunt"],     4
%"discover":["play","pizza","nails","alcatraz","soccer"], 5
%"annoyed":["hat","bmw","pizza","soccer","olives"], 5
%"NRRC":["veggie","boyfriend","muffins","alcatraz","aunt","cupcakes","soccer"], 7
%"know":["play","cheat","hat","alcatraz","aunt","cupcakes","soccer","bmw"], 8
\end{tabular}
\end{center}
\caption{Contents instantiating the projective contents of the 9 triggers in Experiments 1a and 2a. `NRRC' abbreviates `non-restrictive relative clause', `NomApp' abbreviates `nominal appositive' and `possNP' abbreviates `possessive noun phrase'.}\label{t-trigger-content-pairs}
\end{table}

\newpage

The target stimuli were polar questions asked by a speaker, like those in (\ref{target}a) and (\ref{target}b), in which the content `stuntman' instantiates the projective contents of a nominal appositive and of {\em stupid}, respectively:

\begin{exe}
\ex\label{target}
\begin{xlist}
\ex Did Richie, a stuntman, break his leg?
\ex Is Richie stupid to be a stuntman?
\end{xlist}
\end{exe}

The experiment also included 17 control stimuli, which were polar questions formed from the 17 contents in (\ref{contents}): for example, the control stimulus formed from the content `stuntman' is shown in (\ref{control}). The control stimuli were included to confront participants with contents that are neither projective nor at-issue, and to assess whether participants were attending to the task. 

\begin{exe}
\ex\label{control} Is Richie a stuntman?
\end{exe}
The full set of stimuli of Experiment 1a is provided in Appendix \ref{a-exp1a-2a-stimuli}.

Each participant saw a random set of 15 polar questions. Each set contained a target polar question for each of the 9 projective content triggers (each  instantiated by a unique content) and 6 control polar questions (with unique contents as well), for a total of 15 unique contents from (\ref{contents}). Each participant saw their 15 polar questions twice: in one block, participants responded to `certain that' questions to assess projectivity, as shown in (\ref{stim}a). In the other block, participants responded to `asking whether' questions to assess at-issueness, as shown in (\ref{stim}b). Block order and within-block trial order were randomized. Each participant completed 30 trials.


\paragraph{Procedure.} Participants were told to imagine that they are at a party and that, upon walking into the kitchen, they overhear somebody ask another person a question. On each trial, participants read the polar question produced by a random \jd{is it true that speakers were selected randomly?} speaker as well as the corresponding response question, and then gave their response on a slider marked `no' at one end and `yes' at the other, as shown in \figref{f-trial-exp1} for a trial in an `asking whether' at-issueness block.  


\begin{figure}[!h]
\begin{center}
\fbox{\includegraphics[width=12cm]{figures/exp1-trial}}
\end{center}
\caption{A sample (at-issueness) trial in Experiment 1a}\label{f-trial-exp1}
\end{figure}

A `yes' response to a `certain that' question was taken to indicate that the person who uttered the polar question (e.g., Michelle in the sample trial) was committed to the relevant content, i.e., that the content projects, whereas a `no' response was taken to indicate that the content did not project. For the `asking whether' questions, a `yes'  response was taken to indicate that the speaker was asking about the relevant content, i.e., that it was at-issue, whereas a `no' response was taken to indicate that the content was not at-issue. To explore the hypothesis that projectivity and not-at-issueness are positively related,  `yes' responses to `certain that' questions and `no' responses to `asking whether' questions were coded as 1; `no' responses to `certain that' questions and `yes' responses to `asking whether' questions were coded as 0.

After completing the experiment, participants filled out a short optional \jd{is it true that it was optional? it should be, in any case} survey about their age, their native language(s) and, if English is their native language, whether they are a speaker of American English (as opposed to, e.g., Australian or Indian English). To encourage them to respond truthfully, participants were told that they would be paid no matter what answers they gave in the survey.

\paragraph{Data exclusion.}
Prior to analysis, the data from 29 participants who did not self-identify as native speakers of American English were excluded. Inspection of the response means of the remaining 221 participants to the `certain that' and `asking whether' questions for the control polar questions revealed 11 participants whose response means were more than 3 standard deviations above the group means (which were 0.07 for `certain that' and 0.04 for `asking whether' questions). Closer inspection revealed that these participants' responses to the control polar questions were systematically higher than the group means and involved 16 of the 17 contents, suggesting that these participants did not attend to the task or interpreted the task differently. \jd{someone might object that this means we're excluding all the `reasonable' people who get strongly projective readings, and are thereby stacking the deck in our favor (i.e., in favor of finding variability). Or am I reading this the wrong way around and this is before inverting the scale?}The data from these 11 participants were also excluded, leaving data from 210 participants (ages 19-68; median: 33).  


\subsubsection{Results}

We begin by addressing the two main questions of interest: first, \textbf{is there projection variability across and within triggers}? Second, \textbf{is projectivity a function of at-issueness}, as hypothesized by \citet{brst-salt10}?  We addressed these questions by conducting a single mixed-effects linear regression predicting projectivity rating from a centered fixed effect of at-issueness rating. In order to control for block order effects, the model also included a centered fixed effect of block order. Finally, the model included the maximal random effects structure justified by the data and the theoretical questions: random by-trigger intercepts (capturing differences in projectivity between triggers),  random by-content intercepts (capturing differences in projectivity between contents), random by-participant intercepts (capturing individual variability in projectivity), and random slopes for at-issueness by trigger, content, and participant (capturing that the effect of at-issueness may vary across triggers, contents, and participants).  P-values were obtained by remove-one model comparison. The analysis was only conducted on non-main-clause trials, since we are specifically interested in variability in projectivity for cases that even have the potential to project.

\begin{figure}[!h]
\centering
%\begin{center}
\subfloat[][Projectivity means by participant. Error bars indicate bootstrapped 95\% confidence intervals.]{ 
	\includegraphics[width=12cm]{../results/exp1a/graphs/projection-subjectmeans}
	\label{fig:proj-subjmeans}
	}
	
\subfloat[][Boxplot of projection variability by trigger, collapsing across individual contents. Blue dots indicate trigger means, notch indicates median.]{ 
	\includegraphics[width=12cm]{../results/exp1a/graphs/boxplot-projection}
	\label{fig:proj-triggmeans}
}
%\end{center}
\label{fig:f-proj-1a}
\end{figure}

Projection variability would be evidenced by any of the included effects reaching significance at the .05 level. If at-issueness predicts projectivity, this would constitute support for \citetpos{brst-salt10} proposal. If there is significant by-trigger variability in projectivity or by-trigger variability in the effect of at-issueness on projectivity, this would suggest that there are additional, trigger-specific, conventional, effects on projectivity. 


\paragraph{Projection variability} Projection variability is visualized in \figref{fig:f-proj-1a}. About one third of participants took the projective contents to be robustly projective, as can be seen in \figref{fig:proj-subjmeans}. For the remaining participants, the decreasing means reveal a decrease in the overall projectivity of the 9 projective contents and the increasingly larger error bars reveal an increase in projection variability among the 9 projective contents. By-trigger variability can be seen in \figref{fig:proj-triggmeans}:  while median projectivity ratings were all close to ceiling (suggesting that for each projective content at least half of the participants took it to be robustly projective), there was nevertheless variability in projectivity across triggers. For example, mean projectivity of contents paired with \emph{only} was relatively low at .76, while mean projectivity of contents paired with non-restrictive relative clauses and \emph{annoyed} was close to ceiling at .96. 




\paragraph{Influence of at-issueness on projectivity} Mean projectivity ratings are visualized as a function of mean not-at-issueness ratings in \figref{fig:f-proj-ai-1a}. There is a clear relationship between at-issueness and projectivity: the cases that received lower projectivity ratings are also the cases where the projective content was considered by participants to be more at-issue. 

\begin{figure}[!h]

\begin{center}
\includegraphics[width=12cm]{../results/exp1a/graphs/ai-proj-bytrigger-nofacets}

\end{center}

\caption{Mean projectivity against mean not-at-issueness by trigger. Error bars indicate bootstrapped 95\% confidence intervals. }
\label{fig:f-proj-ai-1a}
\end{figure}

These qualitative observations were borne out statistically: there was a significant main effect of at-issueness such that more not-at-issue items received higher projectivity ratings ($\beta$ = 0.37, $SE$ = 0.16, $t$ = 3.48, $p <$ .006). Stepwise model comparison revealed that each random effect was justified, suggesting that there was not only individual variability in projectivity, but also by-trigger variability as well as variability in the at-issueness effect across participants, triggers, and contents. The block effect did not reach significance ($\beta$ = -0.01, $SE$ = 0.01, $t$ = -1.20, $p >$ .23), suggesting that the order in which participants completed the tasks (projectivity, at-issueness) did not affect their judgments in a systematic way. 

For readers who are interested in which triggers differed from each other, we conducted Bonferroni-corrected pairwise comparisons between triggers. \jd{but i really don't think this is interesting.} The results, displayed in \tableref{tab:pairwise}, suggest no difference in projectivity between non-restrictive relative clauses, \emph{annoyed}, nominal appositives, possessive NPs, and \emph{know}. All other trigger projectivities differ from each other (except for the pairs \emph{know/stop}, \emph{stop/discover}, \emph{stop/stupid}, and \emph{discover/stupid}). However, we believe this analysis, in particular the ordering of triggers, is to be taken with a grain of salt, for the following reason: the triggers differed in which contents they occurred with. Given that the regression analysis suggests that there is not only variability by trigger but also by content, a particular trigger's propensity for its projective content to project may be very dependent on the particulars of the content.






\begin{table}

\caption{P-values associated with Bonferroni-corrected paired t-tests on trigger projectivity means.\jd{need to clean up this table if we want to keep it, but too tired right now}}
\begin{tabular}{l l l l l l l l l}
\toprule
 &   NRRC & annoyed & NomApp &  possNP &  know & stop & discover & stupid \\
 \midrule
annoyed &  1.0  &  -  &        -   &       -  &        -  &   - &     -   &  -\\     
NomApp  &  1.0 & 1.0 & -    &   -   &    -    &   -  &     -   & - \\    
possNP  &    1.0 & 1.0 & 1.0 & - &      -  &     -     &  -   & - \\    
know     &   1.0 & 1.0 & 1.0 & 1.0 & -   &    -   &    -       & -\\
stop     &   8.9e-05 & 0.00023 & 0.00099 & 0.01568 & 0.19037 & - &  - & -\\      
discover  &   9.1e-06 & 2.6e-05 & 0.00013 & 0.00265 & 0.04359 & 1.0 & - & -      \\
stupid    &  6.3e-07 & 2.0e-06 & 1.1e-05 & 0.00031 & 0.00719 & 1.0 & 1.0 & - \\
only      &  $<$ 2e-16 & $<$ 2e-16 & $<$ 2e-16 & 1.1e-15 & 3.7e-13 & 2.2e-05 & 0.00020 & 0.00174 \\
\bottomrule
\end{tabular}
\label{tab:pairwise}
\end{table}


 

\begin{figure}[!h]
\begin{center}

\includegraphics[width=12cm]{../results/exp1a/graphs/ai-subjectmeans}

\includegraphics[width=12cm]{../results/exp1a/graphs/boxplot-not-at-issueness}

\end{center}
\caption{At-issueness by participant (top panel) and projective content trigger (bottom panel) \jd{not sure this is an interesting plot}}
\label{f-ai-1a}
\end{figure}




\subsubsection{Discussion}\label{s-discussion1a}

Experiment 1a was designed to explore projection variability for a set of syntactically heterogeneous projective content triggers and the hypothesis that the information-structural status of projective content influences its projectivity. The experiment provides empirical support for both hypotheses. Regarding projection variability, we found both between-participant and between-item variability in how robustly projective content projects. Regarding \citetpos{brst-salt10} projection hypothesis, we found that the information-structural status of projective content, i.e., how not-at-issue it is, is a significant predictor of its projectivity.

Under conventionalist approaches to projection, the projectivity of content is derived from a conventionally specified requirement that the content be entailed by or satisfied in the common ground prior to utterance (e.g., \citealt{heim83,vds92}). Thus, for instance, conventionally requiring that the contents of the complements of {\em discover} and {\em annoyed} are entailed by the common ground of the interlocutors prior to utterance derives the observation that the contents of the complements project over entailment-canceling operators. However, such approaches to projection do not lead us to expect that the contents of the complements of these two predicates differ in how robustly they project since, after all, the two contents receive the same conventional specification. Furthermore, such approaches to projection do not lead us to expect that two speakers of American English will differ in how robustly they take the content of the complement of {\em discover} to project since, again, the content of the complement of {\em discover} is conventionally specified to project for both speakers. Thus, our finding that there is both between-item and between-participant variability in how robustly projective content projects is an empirical challenge to conventional approaches to projection. \jd{i don't think that we can sell the strongest anti-conventionalist story, though, given that not all of the variability can be reduced to at-issueness (which we know from the fact that the by-trigger random effects were justified) -- though it's possible at least in principle that the trigger effects can be reduced to the particular contents they occurred with, we also included by-content random effects that ended up being justified in addition to the trigger random effects. That is, all these things are contributing independent variance.}

Conventionalist approaches are not helped here by appealing to local accommodation, the process by which projective content is accommodated in the scope of an entailment-canceling operator, e.g., to avoid contradictions, uninformativity or problems with binding (\citealt{heim83,vds92}). The context in which participants in Experiment 1a were asked to interpret the polar question stimuli was quite minimal since it only clarified that the participant overheard the speaker uttering the polar question upon entering the kitchen, at a party. Crucially, because the context was so minimal, the relevant projective contents cannot be argued to be locally accommodated to avoid contradictions with the context or to avoid uninformativity; and since the projective contents do not contain variables, problems with binding also do not warrant an appeal to local accommodation. Furthermore, in order for local accommodation to align conventionalist approaches to projection with the observed between-item and -participant variability, one would have to argue that, e.g., the content of the complement of {\em discover} is more likely to be locally accommodated than the content of the complement of {\em annoyed}, or that one participant is more likely to locally accommodate the content of the complement of {\em discover} than another participant. Absent a better understanding of local accommodation, such arguments are implausible. 

One way forward would be to develop a better understanding of local accommodation. Here we take a different route and suggest that the observed between-item and -participant variability provides empirical support for non-conventional approaches to projection, which attempt to derive projectivity from independently-motivated properties of projective content triggers and their use in context (e.g., \citealt{stalnaker74,kempson75,wilson75,boer-lycan76,levinson83,kadmon01,simons01,simons04,atlas05,abusch10,abrusan2011,best-question}). Under such approaches, between-item and -participant variability in projectivity could be attributed to differences between projective contents and their triggers. One such approach, as mentioned above, maintains that projective content differs in its information-structural properties, which has consequences for its projectivity: according to \citealt{brst-salt10} and \citealt{brst-ar}, projective content projects if and only if it is not at-issue. In Experiment 1a, the at-issueness of projective content was explored on the basis of a diagnostic for at-issueness that assumes that at-issue content differs from not at-issue content in how likely it (and its negation) are to partition the context set. The finding of Experiment 1a that the information-structural status of projective content, as diagnosed by this assumption about at-issueness, is a significant predictor of its projectivity provides empirical support the non-conventionalist approach to projection advanced in \citealt{brst-salt10} and \citealt{brst-ar}. In Experiment 1b, we extend our investigation of this hypothesis about projection to a wider range of projective content, namely the contents of the complements of (semi-)factive predicates.


\subsection{Experiment 1b}\label{s-exp1b} 

Experiment 1b was designed to explore the projectivity and at-issueness of the projective contents of the 12 predicates in (\ref{pairs1b2b}), i.e., the contents of the complements of the emotive factives ({\em be amused, be annoyed}), the cognitive factives ({\em know, be aware}), sensory factives ({\em see}), cognitive semi-factives ({\em discover, find out, realize, learn, establish}) and the communication semi-factives ({\em confess, reveal}).

\subsubsection{Methods}

\paragraph{Materials.} The 12 projective contents explored in this experiment were instantiated by the 20 contents shown in (\ref{contents2}). Each of the 12 projective contents was instantiated by each of these 20 contents, for a total of 200 target stimuli. 

\begin{exe}
\ex\label{contents2} 20 contents used in Experiment 1b 

\begin{enumerate}[itemsep=-.5mm]

\begin{multicols}{2}
\item Raul was drinking chamomile tea
\item Jack played frisbee with the kids
\item John was hiding in the garage
\item Mike visited the zoo
\item Zach dyed his hair purple
\item Marissa brought almond cupcakes
\item Chad put up a swing in his backyard
\item Greg drove his car into a ditch
\item Kate fell from her horse
\item Joyce got a poodle 
\columnbreak
\item Carl wrote a poem for his wife
\item Bea posted a family picture on Facebook
\item Janet moved into a damp apartment
\item Samantha bought a fur hat
\item Don ate a chili dog
\item Mary was biting her nails
\item Richie jumped into the pool
\item Martha came in her new BMW
\item Ann was dancing in the corner
\item Sue was doing yoga in the yard
\end{multicols}
\end{enumerate}

\end{exe}

The target stimuli were (past and present tense)\footnote{Stimuli with {\em be amused, be aware} and {\em be annoyed} were realized in the present tense; stimuli with the other predicates were realized in the past tense.} polar questions formed from sentences with one of the 12 predicates, a clausal complement formed from one of the 20 contents in  (\ref{contents2}) and a proper name subject (the names used for the subjects did not occur in the clausal complements). Two sample target stimuli are given in (\ref{sample2}):

\begin{exe}
\ex\label{sample2}
\begin{xlist}
\ex Is Shirley aware that Raul was drinking chamomile tea?

\ex Did Samuel discover that Raul was drinking chamomile tea?
\end{xlist}
\end{exe}

The experiment also included 20 control stimuli, which were (past tense) polar questions formed from sentences conveying the 20 contents in (\ref{contents2}); a sample control polar question is shown in (\ref{sample3}). The control stimuli were included to confront participants with contents that are neither projective nor at-issue, and to assess whether participants were attending to the task.

\begin{exe}
\ex\label{sample3} Was Raul drinking chamomile tea?
\end{exe}

For each participant, a set of 20 polar question stimuli was randomly created: each set contained a target polar question for each of the 12 projective content triggers (each  instantiated by a unique content) and 8 control polar questions (with unique contents as well, for a total of 20 unique contents from (\ref{contents2})). Each participants' 20 polar question stimuli were presented twice: once (in a random order) in the `certain that' block to assess projectivity, and once (in a random order) in the `asking whether' block to assess at-issueness. Thus, each participant responded to questions about 40 polar question stimuli. Block order was random across the participants.

\paragraph{Procedure.} The procedure of Experiment 1b was the same as for Experiment 1a, described in section \ref{s-methods-1a}, except that participants responded to 20 `certain that' and 20 `asking whether' questions. (There are more trials in Experiment 1b than in Experiment 1a because each participant is judging the projectivity and at-issueness of 20 contents rather than 15 contents.)

\paragraph{Participants.} 250 participants with U.S.\ IP addresses and at least 97\% of previous HITs approved were recruited on Amazon's Mechanical Turk platform (ages: 18-74; median: 32). They were paid \$1 for participating in the experiment. 

\subsubsection{Results}

The results about projection variability and the relation between at-issueness and projectivity that are discussed in this section are based on the data from 235 participants (ages 18-74; median: 33). We excluded the data from 3 participants who did not self-identify as native speakers of American English. Inspection of the response means of the remaining 247 American-English speaking participants to the `certain that' and `asking whether' questions for the control polar questions revealed 12 participants whose response means to either were more than 3 standard deviations above the group means (the group means were 0.08 for `certain that' and 0.04 for `asking whether' questions). Further inspection revealed that these participants' responses to the control questions were systematically higher than the group means and involved all of the 20 projective contents, suggesting that these participants did not attend to the task or interpreted the task differently. The data from these 12 participants were also excluded.\footnote{Of the 235 participants, 120 first completed the `asking whether' block and 115 first completed the `certain that' block. \jt{investigate block order effects}}

\paragraph{Projection variability} The upper panel of Figure \ref{f-proj-1b} shows the mean projectivity rating for each of the 235 participants as a blue dot; the error bars indicate 95\% confidence intervals. As with Experiment 1a, we observe variation among the participants in how robustly projective they took the projective contents to be. Compared to Experiment 1a, relatively few participants took the projective contents to be robustly projective. In the lower panel of Figure \ref{f-proj-1b}, the box plots show the response variance for each of the 12 projective content triggers; the blue dot shows the mean projectivity rating. The median projectivity ratings are at ceiling for most of the predicates, with the exception of {\em established, confessed} and {\em revealed}. At the same time, the decreasing mean response and the increasing response variance show that the projectivity of these projective contents is not just subject to participant variability but also to item variability.

\begin{figure}[!h]

\begin{center}
\includegraphics[width=16cm]{../results/exp1b/graphs/projection-subjectmeans}

\includegraphics[width=16cm]{../results/exp1b/graphs/boxplot-projection}

\end{center}
\caption{Projectivity by participant (top panel) and projective content trigger (bottom panel)}
\label{f-proj-1b}
\end{figure}

\jt{Judith Degen insert analysis / projectivity comparison; how do discover and annoyed compare in projectivity here, compared to Exp1a?}


\paragraph{Influence of at-issueness on projectivity} Before examining the influence of at-issueness on projectivity, it is instructive to examine the at-issueness by participant and by projective content trigger. The upper panel of Figure \ref{f-ai-1b} shows the mean at-issueness rating for each of the 235 participants as a blue dot; the error bars again indicate 95\% confidence intervals. Here, about one third of the participants took the projective contents to be robustly not-at-issue. For the remaining participants, the decreasing means reveal a increase in the overall at-issueness of the 12 projective contents. In the lower panel of Figure \ref{f-ai-1b}, the box plots show the response variance for each of the 12 projective content triggers; the blue dot shows the mean at-issueness rating. The median at-issueness ratings are at ceiling, except for {\em established}, suggesting that for each projective content at least half of the participants took it to be robustly not-at-issue. At the same time, the decreasing mean response and the increasing response variances show that the at-issueness of projective content is not just subject to participant variability but also to item variability.

\begin{figure}[!h]
\begin{center}

\includegraphics[width=16cm]{../results/exp1b/graphs/ai-subjectmeans}

\includegraphics[width=16cm]{../results/exp1b/graphs/boxplot-not-at-issueness}

\end{center}
\caption{At-issueness by participant (top panel) and projective content trigger (bottom panel)}
\label{f-ai-1b}
\end{figure}

\jt{How do discover and annoyed compare in not-at-issueness here, compared to Exp1a?}

\jt{Now examine relation between at-issueness and projectivity}

\subsubsection{Discussion}

Like Experiment 1a, Experiment 1b was designed to explore projection variability and the hypothesis that the information-structural status of projective content influences its projectivity (\citealt{brst-salt10,brst-ar}). The findings of Experiment 1b provide further empirical support for projection variability, this time from the (comparatively syntactically homogeneous) set of factive predicates, and thus further substantiate the challenge from projection variability to conventionalist approaches to projection. The findings of the experiment also provide further support for the hypothesis that the information-structural status of projective content influences its projectivity.

\subsection{Summary and discussion of Experiments 1a and 1b}

This paper set out to accomplish two goals, namely to explore projection variability for a broad range of projective content and to test the hypothesis put forth in \citealt{brst-salt10} and \citealt{brst-ar} that the projectivity of projective content is influenced by its information-structural status, specifically its at-issueness. Regarding the first goal, Experiments 1a and 1b have uncovered evidence for projection variability among the projective contents contributed by a set of 19 syntactically heterogeneous projective content triggers. This finding significant expands our understanding of projection variability from previous experimental research (\citealt{xue-onea11,smith-hall11}). As discussed in section \ref{s-discussion1a}, the observed between-participant and -item projection variability presents a challenge to conventionalist approaches to projection. 

Regarding the second goal, Experiments 1a and 1b have provided empirical evidence for the hypothesis that the information-structural status of projective content influences its projectivity (\citealt{brst-salt10,brst-ar}). Such empirical support was already suggested by \citet[180]{xue-onea11}, who, comparing the results of two separate pilot studies, pointed to ``a clear correlation between projection and not-at-issueness''. Since in our Experiments 1a and 1b projectivity and at-issueness was investigated using a within-participant and within-item design, the experimental findings not only provide empirical support for the hypothesis but also allow us to quantify the influence of not-at-issueness on projectivity, and to assess item- and participant-variability.

Our investigation of the second goal critically relied on exploring the at-issueness of projective content. The diagnostic we used for this purpose in Experiments 1a and 1b was chosen for two reasons: first, the diagnostic was used in previous research on at-issueness (e.g., \citealt{amaral-etal07,tonhauser-sula6}), and, second, the diagnostic is applicable for diagnosing at-issueness of projective content realized in polar questions. Since we wanted to explore both projectivity and at-issueness as within-participant and within-item factors, we needed a diagnostic for at-issueness that was suitable. However, a potential worry with the `asking whether' at-issueness diagnostic used in Experiments 1a and 1b is that it may be too closely related to the projection diagnostic. After all, if Patrick, after uttering the polar question in (\ref{stim}), is taken to be certain that Martha's new car is a BMW, then he is presumably not asking whether Martha's new car is a new BMW, and if he is not certain that Martha's new car is a BMW, then he may be more likely to be taken to ask whether Martha's new car is a BMW.\footnote{Of course, according to our hypothesis, projectivity is the flipside of at-issueness.}

\begin{exe}

\exi{(\ref{stim})} Patrick asks: {\em Was Martha's new car, a BMW, expensive?} 

\begin{xlist}
\ex `certain that' question: Is Patrick certain that Martha's new car is a BMW?

\ex `asking whether' question: Is Patrick asking whether Martha's new car is a BMW?

\end{xlist}

\end{exe}
Luckily, many different diagnostics have been used in the literature to diagnose at-issueness (e.g., \citealt{tonhauser-sula6}; for further discussion, see section \ref{s5}). In the second pair of experiments, Experiments 2a and 2b, explore the at-issueness of the projective contents of the first pair of experiments, Experiments 1a and 1b, respectively, using a different diagnostic for at-issueness, with the goal of thereby strengthening our findings about the second goal of the paper. 


\section{Confirming the influence of information structure on projectivity}\label{s4}

In the two experiments discussed in this section, Experiments 2a and 2b, the at-issueness of the projective contents explored in Experiments 1a and 1b, respectively, is explored using a different diagnostic for at-issueness. Whereas the diagnostic used in Experiments 1a and 1b relied on the assumption that the context set is more likely to be partitioned by at-issue content and its negation rather than not-at-issue content and its negation, the diagnostic used in Experiments 2a and 2b relies on the assumption that at-issue content is more likely to be the antecedent of a propositional anaphor than not-at-issue content (see, e.g., \citealt[54]{potts05} and \citealt{tonhauser-sula6}). The 3-turn discourse in (\ref{sure}) illustrates the diagnostic for the appositive content of nominal appositives: the speaker of the first turn (Fred) utters an indicative sentence with the relevant expression and the speaker of the second turn (Carla) responds with the question {\em Are you sure?}, which is taken to involve an elided propositional anaphor {\em that}, as in {\em Are you sure about that?}. The speaker of the third turn, who is identical to the speaker of the first turn, utters an indicative sentence in which the content to be diagnosed realizes the content of the complement of {\em sure}. 


\begin{exe}
\ex\label{sure} 
\begin{xlist}
\exi{Fred:} Martha’s new car, a BMW, was expensive.

\exi{Carla:} Are you sure?

\exi{Fred:} Yes, I am sure that Martha's new car is a BMW.
\end{xlist}

%\ex
%
%\begin{xlist}
%\exi{Sandra:} Billy discovered that Martha has a new BMW.
%
%\exi{Carl:} Are you sure?
%
%\exi{Sandra:} Yes, I am sure that Martha's new car is a BMW.
%\end{xlist}
%\end{xlist}
\end{exe}
To diagnose whether the elided propositional anaphor {\em that} in the second turn can be taken to target the relevant content, participants were asked whether the speaker of the first/third turn answered the question of the other speaker, e.g., in (\ref{sure}) whether Fred answered Carla's question. The assumption is that participants will take Fred to have responded to Carla's question if the relevant content can be taken to be the antecedent of the elided propositional anaphor in Carla's question, i.e., can be at-issue, and participants will take Fred to not have responded to Carla's question if the relevant content cannot be taken to be the antecedent of the elided propositional anaphor in  Carla's question, i.e., cannot be at-issue.

Thus, the at-issueness diagnostic used in Experiments 2a/2b differs from the one used in Experiments 1a/1b in several ways: i) the projective content triggers are realized unembedded in assertions rather than in polar questions; ii) the diagnostic relies on the assumption that at-issue and not-at-issue content differ in the extent to which it can be the antecedent of a propositional anaphor, rather than in the extent to which it and its negation partition the context set; and iii) participants were asked to judge the extent to which an utterance answered a question, rather than what a speaker is asking about.

If the information-structural status of projective content influences its projectivity, we expect the extent to which projective content is not-at-issue under this second diagnostic for at-issueness to also influence the projectivity of the content.

\subsection{Experiment 2a}

Experiment 2a explored the at-issueness of the 9 projective contents that were already explored in Experiment 1a (see section \ref{s-exp1a}) using the {\em Are you sure?} diagnostic, i.e., the content of NRRCs and nominal appositives, the possession implication of possessive noun phrases, the prejacent of {\em only} and {\em stupid}, the pre-state implication of {\em stop} and the contents of the complements of {\em annoyed, discover} and {\em know}.

\subsubsection{Methods}\label{s-methods-2a}

\paragraph{Materials.} Stimuli consisted of 3-turn discourses between two individuals, as (\ref{sure}). In the target stimuli, the first turn of each discourse consisted of an indicative sentence that realized one of the 9 projective content triggers. The 9 projective contents explored in Experiment 2a were instantiated by the same 17 contents as in Experiment 1a, as described in section \ref{s-methods-1a}. Thus, there were a total of 43 indicative sentences with projective content triggers that realized the first turn of the target stimuli. The second turn of the target stimuli consisted of a second speaker's {\em Are you sure?} question. In the third turn, the first speaker utters {\em Yes, I am sure that}, with the relevant projective content realized as the content of the complement of {\em sure}. 

As in Experiment 1a, an additional 17 control discourses were formed with sentences that realize the 17 contents, as shown in (\ref{sure2}).

\begin{exe}
\ex\label{sure2}
\begin{xlist}
\exi{Sandra:} Martha has a new BMW.

\exi{Carl:} Are you sure?

\exi{Sandra:} Yes, I am sure that Martha has a new BMW.
\end{xlist}
\end{exe}
As in Experiment 1a, a set of 15 stimuli was randomly created for each participant: each set contained a target stimulus for each of the 9 projective content triggers (each  instantiated by a unique content) and 6 control stimuli (with unique contents as well, for a total of 15 unique contents). The full set of stimuli of Experiment 2a is provided in Appendix \ref{a-exp1a-2a-stimuli}.

\paragraph{Procedure.} Participants were told to imagine that they are at a party and, upon walking into the kitchen, overhear a short conversation between two people. Participants were then presented with their 15 stimuli in random order and were asked to assess, for each stimulus, whether the speaker of the first/third turn answered the question of the speaker of the second turn. Participants gave their responses on a slider marked `no' at one end and `yes' at the other, as shown in Figure \ref{f-trial-exp2a}. A `yes' response was taken to indicate that the relevant content was taken to be the antecedent of the elided propositional anaphor, i.e., that the content was at-issue; a `no' response was taken to indicate that the relevant content was not taken to be the antecedent of the elided propositional anaphor, i.e., that the content was not at-issue.
After responding to the 15 stimuli, participants completed the same survey as in Experiment 1a.


\begin{figure}[!h]
\begin{center}
\fbox{\includegraphics[width=12cm]{figures/exp2-trial}}
\end{center}
\caption{A sample trial in Experiment 2a}
\label{f-trial-exp2a}
\end{figure}

\paragraph{Participants.} 250 participants with U.S.\ IP addresses and at least 97\% of previous HITs approved were recruited on Amazon's Mechanical Turk platform (ages: 20-77; median: 30). They were paid 30 cents for their participation.

\subsubsection{Results}

The results about the relation between at-issueness and projectivity that are discussed in this section are based on the data from 238 participants (ages 20-77; median: 30). We excluded the data from 6 participants who did not self-identify as native speakers of American English. Inspection of the response means of the remaining 244 American-English speaking participants to the 6 control stimuli revealed 6 participants whose response means were more than 3 standard deviations above the group mean (which was 0.04). Further inspection revealed that these participants' responses were systematically higher than the group mean and involved 14 of the 17 contents, suggesting that these participants did not attend to the task or interpreted the task differently. The data from these 6 participants were also excluded.

\jt{analysis needed}


\begin{figure}[!h]
\begin{center}

\includegraphics[width=12cm]{../results/exp2a/graphs/ai-subjectmeans}

\includegraphics[width=12cm]{../results/exp2a/graphs/boxplot-not-at-issueness}

\end{center}
\caption{At-issueness by participant (top panel) and projective content trigger (bottom panel)}
\label{f-ai-2a}
\end{figure}



\subsubsection{Discussion}

\subsection{Experiment 2b} 

Experiment 2b was designed to explore, using the {\em Are you sure?} diagnostic, the at-issueness of the projective contents of the 12 predicates explored in Experiment 1b, i.e., the emotive factives {\em be amused} and {\em be annoyed}, the cognitive factives {\em know} and {\em be aware}, the sensory factive {\em see}, the cognitive semi-factives {\em discover, find out, realize, learn} and {\em establish} and the communication semi-factives {\em confess} and {\em reveal}.

\subsubsection{Methods}


\paragraph{Materials.} As in Experiment 2a, the stimuli consisted of 3-turn discourses between two individuals. In the target stimuli, the first turn of each discourse consisted of an indicative sentence that realized one of the 12 predicates, as shown in (\ref{sure3}). The contents of the complements of these predicates were instantiated by the same 20 contents as in Experiment 1b, as described in section \ref{s-methods-2a}. Thus, there were a total of 240 indicative sentences with projective content triggers that realized the first turn of the target stimuli. The second turn of the target stimuli consisted of a second speaker's {\em Are you sure?} question. In the third turn, the first speaker utters {\em Yes, I am sure that}, with the relevant projective content realized as the content of the complement of {\em sure}. 

\begin{exe}
\ex\label{sure3}
\begin{xlist}
\exi{Sandra:} Shirley is aware that Raul was drinking chamomile tea.

\exi{Carl:} Are you sure?

\exi{Sandra:} Yes, I am sure that Raul was drinking chamomile tea.
\end{xlist}
\end{exe}

As in Experiment 1b, an additional 20 control discourses were formed with sentences that realize the 20 contents, as shown in (\ref{sure4}).

\begin{exe}
\ex\label{sure4}
\begin{xlist}
\exi{Sandra:} Raul was drinking chamomile tea.

\exi{Carl:} Are you sure?

\exi{Sandra:} Yes, I am sure that Raul was drinking chamomile tea.
\end{xlist}
\end{exe}
As in Experiment 1b, a set of 20 stimuli was randomly created for each participant: each set contained a target stimulus for each of the 12 predicates (whose complements each was instantiated by a unique content) and 8 control stimuli (with unique contents as well, for a total of 20 unique contents). 

\paragraph{Procedure.} The procedure was the same as in Experiment 2a, described in section \ref{s-methods-2a}, except that participants responded to 20 stimuli, rather than just 15.

\paragraph{Participants.} 250 participants with U.S.\ IP addresses and at least 97\% of previous HITs approved were recruited on Amazon's Mechanical Turk platform (ages: 18-77; median: 29). They were paid 30 cents for their participation.

\subsubsection{Results}

The results about the relation between at-issueness and projectivity that are discussed in this section are based on the data from 238 participants (ages 18-77; median: 30). We excluded the data from 6 participants who did not self-identify as native speakers of American English. Inspection of the response means of the remaining 244 American-English speaking participants to the 8 control stimuli revealed 6 participants whose response means were more than 3 standard deviations above the group mean (which was 0.05). Further inspection revealed that these participants' responses were systematically higher than the group mean and involved 18 of the 20 contents, suggesting that these participants did not attend to the task or interpreted the task differently. The data from these 6 participants were also excluded.

\jt{analyses needed: at-issueness variability, influence of at-issueness on projectivity from Exp 1b}

\begin{figure}[!h]
\begin{center}

\includegraphics[width=16cm]{../results/exp2b/graphs/ai-subjectmeans}

\includegraphics[width=16cm]{../results/exp2b/graphs/boxplot-not-at-issueness}

\end{center}
\caption{At-issueness by participant (top panel) and projective content trigger (bottom panel)}
\label{f-ai-2b}
\end{figure}

\subsubsection{Discussion}

\subsection{Summary and discussion of Experiments 2a and 2b}

\jt{analyses needed: relation between the at-issueness diagnostics in Experiments 1a/1b and Experiments 2a/2b}

\section{Discussion}\label{s5}

At-issueness (DGfS talk, \citealt{tonhauser-dgfs2017}):

\begin{itemize}

\item At least 5 different assumptions are made about at-issueness in at-issueness diagnostics that are currently being used in the literature.

\item We have used two diagnostics here that rely on distinct assumptions about at-issueness (partition, anaphoricity). Both revealed an influence of at-issueness on projectivity.

\item But we have also observed that the two diagnostics lead to slightly different findings about at-issueness:

\begin{itemize}

\item e.g., {\em discover} versus {\em stop} (others?)

\item generally higher at-issueness in `are you sure?' diagnostic than in `asking whether' diagnostic suggests that projective content may be more available as antecedent to anaphor than be able to partition the context set

\end{itemize}

\item Furthermore, in a different experimental investigation, \citealt{syrett-koev2015} investigate the at-issueness of NRRCs and nominal appositives using the assent/dissent assumption (though that also involves propositional anaphora). They find that at-issueness depends on position and type. But just numerically, their results are more like our `are you sure?' diagnostic than our `asking whether' diagnostic.

\item Clearly an important topic for future research: i) what's a formal characterization of at-issueness? ii) which properties of at-issue and not-at-issue content derive from this formal characterization? iii) how can these properties be diagnosed with theoretically untrained speakers? 

\end{itemize}


\section{Conclusions}\label{s6}

\appendix

\section{Stimuli used in Experiments 1a and 2a}\label{a-exp1a-2a-stimuli}

The stimuli used in Experiments 1a are grouped here by the 17 instantiating contents. For each content, the first line provides the identifier of the content (e.g., `muffins', for the first content). The second line (`Content') identifies the content that instantiated the projective contents of the relevant projective content triggers. The remaining lines of each of the 17 contents identify the projective content triggers that were instantiated by the content (e.g., `muffins' instantiated control stimuli, NRRCs and {\em only}). In Experiment 2a, indicative sentence variants of the polar questions were used.

\begin{enumerate}

\item  muffins:  \\
     Content: these muffins have blueberries in them\\
     Control stimulus: Do these muffins have blueberries in them?\\
     NRRC: Are these muffins, which have blueberries in them, gluten-free and low-fat?\\
     {\em only}: Do these muffins only have blueberries in them?

\item pizza:  \\
     Content: this pizza has mushrooms on it\\
     Control stimulus: Does this pizza have mushrooms on it?\\
     {\em only}: Does this pizza only have mushrooms on it?\\
     {\em annoyed}: Is Sam annoyed that this pizza has mushrooms on it?\\
     {\em discover}: Did Sam discover that this pizza has mushrooms on it?

\item play:  \\
     Content: Jack was playing outside with the kids\\
     Control stimulus: Was Jack playing outside with the kids?\\
     {\em stop}: Did Jack stop playing outside with the kids?\\
     {\em know}: Does Daria know that Jack was playing outside with the kids?\\
     {\em discover}: Did Paula discover that Jack was playing outside with the kids?

\item veggie:  \\
     Content: Don is a vegetarian\\
     Nominal appositive: Is Don, a vegetarian, going to find something to eat here?\\
     NRRC: Is Don, who is a vegetarian, going to find something to eat here?\\
     Control stimulus: Is Don a vegetarian?

\item cheat:  \\
     Content: Raul cheated on his wife\\
     Control stimulus: Did Raul cheat on his wife?\\
     {\em know}: Does Daria know that Raul cheated on his wife?\\
     {\em stupid}: Was Raul stupid to cheat on his wife?

\item nails:  \\
     Content: Mary's daughter has been biting her nails\\
     Control stimulus: Has Mary's daughter been biting her nails?\\
     {\em discover}: Did Mary discover that her daughter has been biting her nails?\\
     {\em stop}: Has Mary's daughter stopped biting her nails?\\
     {\em stupid}: Is Mary's daughter stupid to be biting her nails?

\item  ballet:  \\
     Content: Ann used to dance ballet\\
     Control stimulus: Did Ann use to dance ballet?\\
     Nominal appositive: Is Ann, a former ballet dancer, limping?\\
     {\em stop}: Did Ann stop dancing ballet?

\item kids:  \\
     Content: John's kids were in the garage\\
     {\em only}: Were John's kids only in the garage?\\
     Control stimulus: Were John's kids in the garage?\\
     {\em stupid}: Were John's kids stupid to be in the garage?

\item hat:  \\
     Content: Samantha has a new hat\\
     Control stimulus: Does Samantha have a new hat?\\
     Possessive NP: Was Samantha's new hat expensive?\\
     {\em know}: Does Daria know that Samantha has a new hat?\\
     {\em annoyed}: Is Joyce annoyed that Samantha has a new hat?

\item bmw:  \\
     Content: Martha has a new BMW\\
     Control stimulus: Does Martha have a new BMW?\\
     Possessive NP: Was Martha's new BMW expensive?\\
     Nominal appositive: Was Martha's new car, a BMW, expensive?\\
     {\em annoyed}: Is Martha's neighbor annoyed that Martha has a new BMW?\\
     {\em know}: Does Billy know that Martha has a new BMW?

\item boyfriend:  \\
     Content: Betsy has a boyfriend\\
     Control stimulus: Does Betsy have a boyfriend?\\
     NRRC: Is Betsy, who has a boyfriend, flirting with the neighbor?\\
     Possessive NP: Is Betsy's boyfriend from around here?

\item alcatraz:  \\
     Content: Mike visited Alcatraz\\
     Control stimulus: Did Mike visit Alcatraz?\\
     NRRC: Is Mike, who visited Alcatraz, a history fan?\\
     {\em discover}: Did Jane discover that Mike visited Alcatraz?\\
     {\em know}: Does Jane know that Mike visited Alcatraz?

\item aunt:  \\
     Content: Janet has a sick aunt\\
     Control stimulus: Does Janet have a sick aunt?\\
     NRRC: Is Janet, who has a sick aunt, very compassionate?\\
     {\em know}: Does Melissa know that Janet has a sick aunt?\\
     Possessive NP: Has Janet's sick aunt been recovering?

\item cupcakes:  \\
     Content: Marissa brought the cupcakes\\
     Control stimulus: Did Marissa bring the cupcakes?\\
     NRRC: Is Marissa, who brought the cupcakes, a good baker?\\
     {\em know}: Does Max know that Marissa brought the cupcakes?

\item soccer:  \\
     Content: the soccer ball has a hole in it\\
     Control stimulus: Does the soccer ball have a hole in it?\\
     NRRC: Was the soccer ball, which has a hole in it, a gift from Uncle Bill?\\
     {\em annoyed}: Is Mandy annoyed that the soccer ball has a hole in it?\\
     {\em discover}: Did Mandy discover that the soccer ball has a hole in it?\\
     {\em know}: Does Mandy know that the soccer ball has a hole in it?

\item olives:  \\
   	Content: this bread has olives in it\\
   	Control stimulus: Does this bread have olives in it?\\
   	{\em annoyed}: Is Barbara annoyed that this bread has olives in it?

\item stuntman:  \\
   	Content: Richie is a stuntman\\
   	Control stimulus: Is Richie a stuntman?\\
   	Nominal appositive: Did Richie, a stuntman, break his leg?\\
   	{\em stupid}: Is Richie stupid to be a stuntman?

\end{enumerate}

\bibliographystyle{cslipubs-natbib}
\bibliography{bibliography}


\end{document}
