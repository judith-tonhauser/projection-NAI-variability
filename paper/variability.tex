
\documentclass[11pt,fleqn]{article}
\usepackage[margin=1in,top=1in,bottom=1in]{geometry}
\usepackage{mathtools}
\usepackage{longtable}
\usepackage{enumitem}
\usepackage{hyperref}
\usepackage[dvips]{graphics}
\usepackage[table]{xcolor}
\usepackage{amssymb}
\usepackage{subfig}
\usepackage{booktabs}

\usepackage[normalem]{ulem}

\usepackage{multicol}
\usepackage{txfonts}
%\usepackage{amsfonts}
\usepackage{natbib}
\usepackage{gb4e}
%\usepackage{/Users/judith/Library/Latex/drs}
%\usepackage{/Users/judith/Library/Latex/avm}
\usepackage[all]{xy}
\usepackage{rotating}
\usepackage{tipa}
\usepackage{multirow}
\usepackage{authblk}
\usepackage{adjustbox}
\usepackage{array}

\newcolumntype{R}[2]{%
    >{\adjustbox{angle=#1,lap=\width-(#2)}\bgroup}%
    l%
    <{\egroup}%
}
\newcommand*\rot{\multicolumn{1}{R{60}{1em}}}% no optional argument here, please!

\newcommand{\foc}{$_{\mbox{\small F}}$}
\newcommand{\lp}{<_{\hspace*{-.1cm}p}}
\newcommand{\lnai}{<_{\hspace*{-.1cm}nai}}

\setlength{\parindent}{.8cm}
\setlength{\parskip}{0ex}
\setlength{\headsep}{0in}

\setlength{\bibsep}{0mm}
\bibpunct[:]{(}{)}{;}{a}{,}{,}

\newcommand{\yi}{\'{\symbol{16}}}
\newcommand{\nasi}{\~{\symbol{16}}}
\newcommand{\hina}{h\nasi na}
\newcommand{\ina}{\nasi na}
%\renewcommand{\abut}{$\supset$\hspace*{-0.07cm}$\subset$}
\newcommand{\tto}{t$_{top}$}
\newcommand{\wtop}{w$_{top}$}
\newcommand{\tc}{t$_c$}
\newcommand{\schwa}{\begin{sideways}e\end{sideways}}

% Semantic brackets
%\newcommand{\iss}[1]{\mbox{\protect\tiny \mbox{#1}}}
%\newcommand{\sem}[2]{\6#1\9$_\iss{#2}$} David's original
\newcommand{\6}{\mbox{$[\hspace*{-.6mm}[$}} 
\newcommand{\9}{\mbox{$]\hspace*{-.6mm}]$}}
\newcommand{\sem}[2]{\6#1\9$^{#2}$}

\newcommand{\semt}[2]{$\left[\hspace*{-.6mm}\left[\begin{tabular}[c]{@{}l@{}}#1\vspace*{-.5em}\end{tabular}\right]\hspace*{-.6mm}\right]\hspace*{-.6mm}^{#2}$}

%\renewcommand{\baselinestretch}{1.2}

\def\bad{{\leavevmode\llap{*}}}
\def\marginal{{\leavevmode\llap{?}}}
\def\verymarginal{{\leavevmode\llap{??}}}
\def\infelic{{\leavevmode\llap{\#}}}

\definecolor{Lighter}{gray}{.92}
\definecolor{Blue}{RGB}{0,0,255}
\definecolor{Green}{RGB}{10,200,100}
\definecolor{Red}{RGB}{255,0,0}


\newcommand{\citepos}[1]{\citeauthor{#1}'s \citeyear{#1}}
\newcommand{\citeposs}[1]{\citeauthor{#1}'s}
\newcommand{\citetpos}[1]{\citeauthor{#1}'s (\citeyear{#1})}

\newcommand{\eref}[1]{(\ref{#1})}
\newcommand{\tableref}[1]{Table \ref{#1}}
\newcommand{\figref}[1]{Fig.~\ref{#1}}
\newcommand{\appref}[1]{Appendix \ref{#1}}
\newcommand{\sectionref}[1]{Section \ref{#1}}


\title{Information-structure as a source of projection variability}

\author[$\bullet$]{Judith Tonhauser}
\author[$\triangleright$]{Judith Degen}
\author[$\circ$]{David Beaver}

\affil[$\bullet$]{The Ohio State University}
\affil[$\triangleright$]{Stanford University}
\affil[$\circ$]{University of Texas at Austin}

\renewcommand\Authands{ and }

\newcommand{\jt}[1]{\textbf{\color{blue}JT: #1}}
\newcommand{\jd}[1]{\textbf{\color{Green}[jd: #1]}}  

\begin{document}

\maketitle

\begin{abstract}
Projective content is utterance content that a speaker may be taken to be committed to it even when the expression associated with the content occurs embedded under an entailment-canceling operator (e.g., \citealt{ccmg90}). Conventionalist approaches to projection assume that the projectivity of projective content derives from a conventionally specified requirement of such content to be entailed by or satisfied in the common ground of the interlocutors (e.g., \citealt{heim83,vds92}). An empirical challenge for conventionalist approaches comes from the observation that projective content varies in how robustly it projects (e.g., \citealt{karttunen71b,simons01,abusch10}). An alternative approach, advanced in \citealt{brst-salt10}, maintains that utterance content projects if and only if it is not at-issue with respect to the question addressed by the utterance (see also \citealt{brst-ar}). Under this approach, projection variability can be attributed to variable at-issueness. This paper reports the findings of two pairs of experiments designed to explore the robustness with which projective content associated with a wide range of English expressions projects and to test \citetpos{brst-salt10} hypothesis that the information-structural status of projective content is related to projectivity. The two experiments provide robust empirical evidence for projection variability and suggest that both the expression associated with the projective content and the information-structural status of projective content plays a role in its projectivity. The paper concludes by discussing the implications of these findings for an empirically adequate analysis of projection.

\end{abstract}

%\tableofcontents

%\newpage
			
\section{Introduction}\label{s1}

Projective content is utterance content that the speaker may be taken to be committed to even when the expression associated with the content occurs in the syntactic scope of an entailment-canceling operator (see, e.g., \citealt{ccmg90}). To illustrate, consider the examples in (\ref{eng1}) and (\ref{eng2}). The speaker of (\ref{eng1}) is taken to be committed to the content of the complement of {\em discover}, that Mike visited Alcatraz, since this content is entailed by the utterance. The so-called Family-of-Sentences variants of (\ref{eng1}) given in (\ref{eng2}a-d) do not entail this content because {\em discover} is embedded under entailment-canceling operators: negation in (\ref{eng2}a), the polar question operator in (\ref{eng2}b), the epistemic possibility modal {\em perhaps} in (\ref{eng2}c) and the antecedent of a conditional in (\ref{eng2}d). Since speakers who utter the sentences in (\ref{eng2}a-d) may nevertheless be taken to be committed to the content of the complement, this content, by virtue of being able to `project' over the entailment-canceling operators, is considered projective content. 

\begin{exe}
\ex\label{eng1}  Felipe discovered that Mike visited Alcatraz.

\ex\label{eng2}
\begin{xlist} 
\ex Felipe didn't discover that Mike visited Alcatraz.
\ex Did Felipe discover that Mike visited Alcatraz?
\ex Perhaps Felipe discovered that Mike visited Alcatraz.
\ex If Felipe discovered that Mike visited Alcatraz, he'll get mad.
\end{xlist}
\end{exe}

Why does projective content project? According to mainstream approaches, projective content projects because it is conventionally specified to do so, for instance, by being required to be entailed by or satisfied in the common ground of the interlocutors (e.g., \citealt{heim83,vds92,geurts99}). On such `conventionalist' approaches, the lexical entry of {\em discover} specifies that the content of its clausal complement is required to be entailed by or satisfied in the common ground of the interlocutors, thereby ensuring that the speaker is taken to be committed to the content. Since conventionalist approaches only distinguish projective and non-projective content, such approaches are challenged by the long-standing observation that some projective content projects less robustly than other such content. In the early 1970s already, \citet{karttunen71b} pointed out that the content of the complement of {\em regret} in (\ref{semi-factive}a) is more robustly projective than the content of the complement of {\em discover} in (\ref{semi-factive}b); following \citealt{karttunen71b}, predicates like {\em discover} have been referred to as `semi-factive', in contrast to their `factive' counterparts like {\em regret} (and `non-factive' predicates like {\em believe}). \citet{schlenker10} referred to the predicate {\em announce} as a `part-time trigger' because the content of its complement may, but often does not, project.

\begin{exe}
\ex\label{semi-factive}
\begin{xlist}
\ex John didn't discover that he had not told the truth.  
\ex John didn't regret that he had not told the truth.
\hfill (\citealt[63]{karttunen71b})

\end{xlist}
\end{exe}

Similarly, \citet[432]{simons01} noted, partially based on examples from \citealt{ccmg90} and \citealt{geurts94}, that the projection of ``some -- but crucially, not all'' projective content to the common ground of the interlocutors may be suppressed in explicit ignorance contexts. Example (\ref{hardsoft}a), for instance, shows that the projective content of {\em win} in the antecedent of the conditional, that John participated in the race, need not be part of the common ground of the interlocutors, i.e., need not project. On the other hand, the existential implication of the cleft in (\ref{hardsoft}b), that there is an individual who read the letter, must be part of the common ground of the interlocutors, i.e., must project. Expressions like {\em win} are referred to as `soft triggers', in contrast to `hard triggers', like the cleft (see also, e.g., \citealt{abusch10}).\footnote{In this paper, we do not use the term `trigger' to refer to expressions associated with projective content since this term evokes conventional theories of projection. To remain neutral about how projective content comes to be projective, we instead use `expression associated with projective content'.}

\begin{exe}
\ex\label{hardsoft}
\begin{xlist}

\ex I have no idea whether John ended up participating in the Road Race yesterday. But if he won it, then he has more victories than anyone else in history. \hfill (\citealt[39]{abusch10})

\ex\infelic I have no idea whether anyone read that letter. But if it is John
who read it, let's ask him to be discreet about the content. \hfill (\citealt[40]{abusch10})

\end{xlist}
\end{exe}

Experimental research has provided preliminary evidence for projection variability. \citet{xue-onea11} observed that the content of the complement of German {\em wissen} `know' is less projective than the content of the complement of {\em erfahren} `find out', both of which are less projective than the relevant projective contents of sentences with {\em auch} `too' (that a parallel event is contextually salient) and {\em wieder} `again' (that the relevant event has happened before). Similarly, \citet{smith-hall11} found that the projective contents of {\em win} and {\em know} are less projective than the content implication of English definite noun phrases (e.g., for {\em the queen}, that the referent is a queen). Interestingly, they also found that the existential implication of cleft sentences, considered a hard trigger, was (numerically) less projective than the relevant contents of the soft triggers {\em win} and {\em know}. Thus, the sparse experimental evidence confirms some but not all of the intuitions about projection variability reported in the literature.\footnote{\citet{tiemann-etal11} noted that projective content differs in how acceptable it was judged in contexts that did not entail the relevant content, but projection variability was not the focus of their paper.}

Observations about projection variability challenge conventionalist approaches to projection because such approaches do not offer an explanation for why some projective content seems to systematically project less robustly than other such content. After all, under such approaches, any expression associated with projective content lexically specifies that the relevant content is required to be entailed by or satisfied in the common ground of the interlocutors. The lexical specifications of expressions like {\em regret, discover, win}, clefts and definite noun phrases predict that their relevant contents can project, but do not predict differences in how robustly the contents project. Referring to expressions associated with projective content as `semi-factive' or `soft', as opposed to `factive' or `hard', serves to remind of projection variability but does not address the challenge that this variability poses for conventionalist approaches to projection.

One possible explanation for projection variability is that projectivity derives from a property that projective content shares, but shares to varying degrees. \citet{brst-salt10} proposed that this property is `at-issueness', i.e., the ability of content to address the Question Under Discussion (QUD; e.g., \citealt{roberts12}).  Specifically, Simons and her colleagues proposed that utterance content projects if and only if it is not at-issue with respect to the QUD addressed by the utterance. This hypothesis was formulated as the Projection Principle in \citealt[280]{brst-ar}:\footnote{According to \citealt[315]{brst-salt10}, there is a causal relation between projection and at-issueness: utterance content projects not just {\em when} but {\em because} it is not-at-issue. The Projection Principle as formulated in \citealt{brst-ar} is neutral about whether not-at-issueness causes projection or whether not-at-issueness is merely correlated with projection. The experiments we report on here were designed to test the Projection Principle, i.e., whether not-at-issueness is correlated with projection. We leave the question of whether there is a causal relationship between not-at-issueness and projection to future research.}

\begin{exe}
\ex\label{pp} {\bf Projection Principle:} If content $C$ is expressed by a constituent embedded under an entailment-canceling operator, then $C$ projects if and only if $C$ is not at-issue.

\end{exe} 
Although Simons and her colleagues did not consider projection variability, the Projection Principle predicts such variability. It predicts, for instance, that the content of non-restrictive relative clauses (NRRCs) projects more robustly, by virtue of being not at-issue (\citealt{potts05}), than the content of the complement of {\em discover}, which can be at-issue and not-at-issue (\citealt{simons07}). Consider the question-answer pairs in (\ref{nrrc}): the example in (\ref{nrrc}a) shows that the content of the NRRC in B's utterance cannot be used to address A's question, and hence is not at-issue; the example in (\ref{nrrc}b) shows that the main clause content of B's utterance can address A's question and hence is at-issue.\footnote{\citet{syrett-koev2015} show that the content of NRRCs can be the target of direct denial, which may be taken to suggest that this content can be at-issue. We return to this matter in section \ref{s-disc2}.}  The content of the complement of {\em discover}, on the other hand, can be not-at-issue, as shown in (\ref{discover}a), but it can also be at-issue, as shown in (\ref{discover}b). Thus, if the content of NRRCs is more robustly not-at-issue than the content of the complement of {\em discover}, the former is predicted by the Projection Principle to project more robustly than the latter. 

\begin{exe}
\ex\label{nrrc}
\begin{xlist}
\ex
\begin{xlist}
\exi{A:} Did Mike visit Alcatraz?
\exi{B:} \infelic Mike, who visited Alcatraz, is interested in the history of prisons.
\end{xlist}


\ex
\begin{xlist}
\exi{A:} What is Mike interested in?
\exi{B:} Mike, who visited Alcatraz, is interested in the history of prisons.
\end{xlist}


\end{xlist}

\ex\label{discover}
\begin{xlist}

\ex
\begin{xlist}
\exi{A:} Why is Henry in such a bad mood?
\exi{B:} He discovered that Harriet had a job interview at Princeton. 
\end{xlist}

\ex
\begin{xlist}
\exi{A:} Where was Harriet yesterday?
\exi{B:} Henry discovered that she had a job interview at Princeton. \hfill (\citealt[1035]{simons07})
\end{xlist}
\end{xlist}
\end{exe}

This paper has two goals, which are explored on the basis of two pairs of experiments. The first goal (Exps.~1a and 1b) is to explore projection variability for a broad range of projective content, to better understand the extent to which projective content varies in  projectivity. The second goal (addressed through Exps.~1a and 1b, as well as Exps.~2a and 2b) is to test the Projection Principle, namely that the at-issueness of projective content is inversely correlated with its projectivity. These two goals jointly serve to identify empirical generalizations that an empirically adequate analysis of projection needs to account for. 


In pursuing our first goal, we expand on and improve on previous experimental research on projection variability. In \citealt{xue-onea11} and \citealt{smith-hall11}, projection variability was explored for 4 German and 6 English expressions associated with projective content, respectively. Exps.~1a and 1b significantly broaden our understanding of projection variability by considering the projective content associated with 19 expressions. Our experiments also take into consideration that lexical content may influence projection: a speaker might, for instance, be more likely to be taken to be committed to the content that Alexander flew to New York than to the content that Alexander flew to the moon, simply because people are more likely to fly to New York than the moon. In other words, the lexical content that instantiates projective content\footnote{\label{f-content}We use the term `lexical content' to refer to the content of an expression and the term `projective content' to refer to an abstract characterization of the projective content of an expression. For instance, in B's utterance in (\ref{discover}a), the relevant expression is {\em discover}, the projective content is the content of its clausal complement, and the lexical content (of the projective content) is that Harriet had a job interview at Princeton.} may matter for how robustly the projective content is taken to be a commitment of the speaker and how likely the projective content is to address the QUD. The projective content of the 6 expressions explored in \citealt{smith-hall11} was only instantiated by one lexical content each and a distinct lexical content instantiated each projective content. Our experiments, in contrast, included a total of 37 lexical contents and the projective content of each expression was instantiated by up to 20 lexical contents. Furthermore, to facilitate comparison across different projective contents and the expressions associated with the projective content, the projective contents of distinct expressions were instantiated with the same lexical contents: overall, each of the 37 lexical contents instantiated up to 12 projective contents.

Our second goal -- to test the Projection Principle -- also builds on and significantly expands previous experimental work. 
 Using a direct dissent diagnostic for at-issueness, \citet{amaral-etal11} found that speakers of British English judged direct dissent with the projective content of {\em only} (the prejacent) to be more acceptable than direct dissent with the projective content of {\em continue} and {\em stop} (the pre- and post-state implications, respectively). These findings suggest that the prejacent of {\em only} is more at-issue than the post- and pre-state implications of {\em continue} and {\em stop}, respectively. (See also \citealt{cummins-etal2012}, and \citealt{amaral-cummins2015} for similar results on Spanish.) \citet{xue-onea11} found that speakers of German were more likely to directly dissent with the content of the complement of {\em wissen} `know' than with the content of the complement of {\em erfahren} `find out' and, in turn, more likely to directly dissent with these contents than with projective contents of {\em auch} `too' and {\em wieder} again'. These results suggest that the projective content of {\em wissen} `know' is more at-issue than the projective content of {\em erfahren} `find out', which in turn is comparatively more at-issue than the projective contents of {\em auch} `too' and {\em wieder} again'. Interestingly, comparing the relative projectivity and not-at-issueness 
of the projective contents across their two experiments, \citet[180]{xue-onea11} point to ``a clear correlation between projection and not-at-issueness'', in line with the Projection Principle. Our Exps.~1a and 1b improve on \citetpos{xue-onea11} study by exploring the projectivity and at-issueness of projective content as within-item and within-participant factors. The design of these experiments therefore allows us to quantify the correlation between not-at-issueness and projectivity, and to consider by-item and by-participant variability. Since different diagnostics have been used in the literature to diagnose at-issueness (for discussion, see, e.g., \citealt{tonhauser-sula6} and section \ref{s-summary1a1b}), the second pair of experiments, Exps.~2a and 2b, explore the at-issueness of the projective contents of the first pair of experiments, Exps.~1a and 1b, using a different diagnostic for at-issueness to make sure that the at-issueness results in Exps.~1 are not just an artifact of the at-issueness diagnostic used.

The paper proceeds as follows. In section \ref{s2}, we characterize the projective contents explored in this paper. Section \ref{s3} addresses the two goals of the paper on the basis of Exps.~1a and 1b, and section \ref{s4} extends our investigation of the second goal on the basis of Exps.~2a and 2b. In section \ref{s5}, we discuss the implications of our findings for analyses of projection. Section \ref{s6} concludes the paper.


\section{The projective contents explored in the experiments}\label{s2}

The projective contents associated with 19 target expressions are explored in the two pairs of experiments. The projective content associated with the 9 target expressions in Exp.~1a are syntactically heterogeneous, as shown in (\ref{pairs1a2a}), whereas the 12 target expressions included in Exp.~1b are all (semi-)factive attitude predicates, as shown in (\ref{pairs1b2b}). Since some expressions can contribute multiple projective contents (\citealt{brst-lang11}), we specify, for each target expression, the projective content investigated in our experiments: e.g., in (\ref{pairs1a2a}a), `sentence-medial NRRCs' identifies the target expression and `content of the NRRC' identifies the projective content. For the syntactically heterogeneous target expressions in (\ref{pairs1a2a}), a sample sentence illustrating the expression is provided and the corresponding projective content is identified. Exps.~2a and 2b explored not-at-issueness for the same expression/projective content pairs as Exps.~1a and 1b, respectively. Since the projective content of the attitude predicates {\em is annoyed} and {\em discovered} was explored in both pairs of experiments, a total of 19 expression/projective content pairs were explored. 

\begin{exe}
\ex\label{pairs1a2a} {\bf Target expression / projective content pairs in Experiments 1a and 2a}

\begin{enumerate}[itemsep=-.5mm]

\item Sentence-medial NRRCs / content of the NRRC
\\ e.g., {\em These muffins, which have blueberries in them, are gluten-free and low-fat.} / `These muffins have blueberries in them.'

\item Sentence-medial nominal appositives / appositive content
\\ e.g., {\em Martha's new car, a BMW, was expensive.} / `Martha's new car is a BMW'

\item Posessive noun phrases / possession implication
\\ e.g., {\em Martha's new BMW was expensive.} / `Martha's new car is a BMW'

\item {\em be annoyed} / content of the clausal complement
\\ e.g., {\em Martha's neighbor is annoyed that Martha has a new BMW.} / `Martha has a new BMW'

\item {\em discover} / content of the clausal complement
\\ e.g., {\em Mary discovered that her daughter has been biting her nails.} / `Mary's daughter has been biting her nails'

\item {\em know} / content of the clausal complement
\\ e.g., {\em Billy knows that Martha has a new BMW.} /  `Martha has a new BMW'

\item {\em only} / prejacent
\\ e.g., {\em These muffins only have blueberries in them.} / `These muffins have blueberries in them'

\item {\em stop} / pre-state implication
\\ e.g., {\em Mary's daughter stopped biting her nails.}  / `Mary's daughter has been biting her nails'

\item {\em be stupid to} / prejacent
\\ e.g., {\em Mary's daughter is stupid to be biting her nails.} / `Mary's daughter has been biting her nails'

\end{enumerate}


\ex\label{pairs1b2b} {\bf Target expression / projective content pairs in Experiments 1b and 2b}

\begin{enumerate}[itemsep=-.5mm]

\item {\em be amused} / content of the clausal complement

\item {\em be annoyed} / content of the clausal complement

\item {\em know} / content of the clausal complement

\item {\em be aware} / content of the clausal complement

\item {\em see} / content of the clausal complement

\item {\em discover} / content of the clausal complement

\item {\em find out} / content of the clausal complement

\item {\em realize} / content of the clausal complement

\item {\em learn} / content of the clausal complement

\item {\em establish} / content of the clausal complement

\item {\em confess} / content of the clausal complement

\item {\em reveal} / content of the clausal complement

\end{enumerate}

\end{exe}

In what follows, we characterize the properties of the 19 target expression/projective content pairs explored in the experiments.

\paragraph{Projectivity} The 19 projective contents share the property of being projective, i.e., being able to be taken to be a commitment of the speaker even when the expression that the projective content is associated with is embedded under an entailment-canceling operator. The 9 expressions included in Experiments 1a and 2a differ in how robustly their contents have been reported to project: 
the relevant contents of NRRCs, nominal appositives and of the emotive `factive' predicate {\em annoyed} are typically taken to project more robustly than, e.g., the prejacent of {\em only}, the pre-state implication of {\em stop} and the complement of the cognitive `semi-factive' predicate {\em discover} (e.g., \citealt{karttunen71b,simons01,potts05,abusch10,beaver-belly}). The 12 target expressions included in Experiments 1b and 2b include both `factive' and `semi-factive' predicates, and were also chosen to denote different types of attitudes: emotive `factive' predicates ({\em be amused, be annoyed}), cognitive `factive' predicates ({\em know, be aware}), sensory `factives' predicates ({\em see}), cognitive `semi-factives' predicates ({\em discover, find out, realize, learn, establish}) and communication `semi-factives' ({\em confess, reveal}). The predicates {\em be annoyed} and {\em discover} were included in both Exps.~1a and 1b (and 2a and 2b) to be able to directly compare the results of the experiments.

\paragraph{Strong Contextual Felicity} A property shared by the 19 target expression/projective content pairs is that they do not impose a Strong Contextual Felicity constraint on the utterance context (\citealt{brst-lang11}). What this means is that utterances with the target expressions included in the experiments are judged to be acceptable in contexts in which the projective content is not already part of the common ground of the interlocutors when the expression is uttered. For instance, B's utterance in (\ref{nrrc}b), repeated below, is acceptable even if A did not previously know the content of the NRRC, that Mike visited Alcatraz. An expression/projective content pair that is associated with a Strong Contextual Felicity contraint is the pronoun {\em they} and the projective content that there is a uniquely salient plurality of individuals (to which the pronoun refers). Use of {\em they} in (\ref{scf}) is judged to be unacceptable because the projective content of the pronoun is not part of the common ground of the interlocutors, i.e., the utterance context does not entail the existence of a uniquely salient plurality of individuals to which the pronoun could refer. 
%Other projective content triggers that impose a Strong Contextual Felicity constraint include {\em too} (there is a salient alternative proposition) and the deictic implication of demonstratives (see \citealt{brst-lang11}).

\begin{exe}
\exi{(\ref{nrrc}b)}
\begin{xlist}
\exi{A:} What is Mike interested in?
\exi{B:} Mike, who visited Alcatraz, is interested in the history of prisons.
\end{xlist}

\ex\label{scf} At a bus stop, one woman asks another one, with no other people around: \\ \infelic Did they visit Alcatraz?
\end{exe}

Including in our experiments only target expression/projective content pairs not associated with a Strong Contextual Felicity constraint was motivated by our goal of exploring the relative projectivity and not-at-issueness of the projective contents associated with the target expressions. It is well-known that the context in which an expression associated with a projective content occurs influences whether the projective content projects (see, e.g., the examples in (\ref{hardsoft}), but also, e.g., \citealt{simons01,beaver-belly}). In our experiments, all of the target expressions were therefore presented in the same contexts, namely ones that clarified the situation in which the expression was uttered but that minimized the extent to which the context might influence the projectivity or not-at-issueness of the relevant content. Including target expression/projective content pairs associated with a Strong Contextual Felicity constraint would have forced us to present the expressions in different contexts, thereby hampering comparison across expressions.\footnote{As discussed in section \ref{s5}, the minimal contexts in which the stimuli were presented also have the property of not plausibly licensing `local accommodation' (\citealt{heim83,vds92}), the process that allows conventionalist approaches to projection to account for projective content not projecting.} 

\paragraph{Obligatory Local Effect} The 19 target expression/projective content pairs differ in whether they have Obligatory Local Effect (\citealt{brst-lang11}), the property that distinguishes target expression/projective content pairs based on whether the projective content is obligatorily contributed to an attitude holder's belief state when the expression occurs in the complement of a belief-predicate. The examples in (\ref{ole}) illustrate that the content of NRRCs does not have Obligatory Local Effect, in contrast to the content of the complement of {\em discover}, which does. In (\ref{ole}a), where the NRRC occurs in the complement clause of {\em believe}, the attitude holder (Sarah) need not be committed to the content of the NRRC, that Mike visited Alcatraz, as shown by the acceptability of the continuation. In contrast, in (\ref{ole}b), where {\em discover} occurs in the complement of the belief-predicate, Sarah must be committed to the same content (that Mike visited Alcatraz), as shown by the unacceptability of the same continuation. 

\begin{exe}
\ex\label{ole}
\begin{xlist}
\ex Sarah believes that Mike, who visited Alcatraz, is interested in the history of prisons, \ldots but she doesn't know that Mike has visited Alcatraz.

\ex Sarah believes that Felipe discovered that Mike visited Alcatraz, \ldots \#but she doesn't know that Mike has visited Alcatraz. 

\end{xlist}
\end{exe}
In addition to the projective content of NRRCs, the projective contents of appositives and of possessive noun phrases do not have Obligatory Local Effect. The remaining 16 target expression/projective content pairs, including that of {\em discover}, have Obligatory Local Effect. 

In sum, the 9 target expressions explored in Experiments 1a and 2a are syntactically heterogeneous, have been reported to differ in how robustly the relevant projective content projects and include 3 that have Obligatory Local Effect and 6 that don't. The 12 target expressions explored in Experiments 1b and 2b are all attitude predicates, i.e., syntactically comparatively homogenous, and all have Obligatory Local Effect, but also differ in how robustly their projective contents have been reported to project.


\section{Experiment 1}
\label{s3}

Exps.~1a and 1b were designed to explore the research questions in (\ref{questions}) for the 19 projective contents introduced in the previous section.

\begin{exe}
\ex\label{questions} {\bf Research questions}

\begin{xlist} 

\ex Does projective content vary in how robustly it projects?

\ex Is projectivity a function of at-issueness, as predicted by the Projection Principle?
\end{xlist}

\end{exe} 
To explore these research questions, we collected judgments about both the projectivity and the at-issueness of the 19 projective contents in sentences in which the target expressions were embedded in polar questions. Consider, for instance, Patrick's polar question with the nominal appositive in (\ref{stim}). The relevant projective content here is that Martha's new car is a BMW. The `certain that' response question in (\ref{stim}a) assesses the extent to which Patrick is taken to be committed to the projective content, i.e., the extent to which the content projects from Patrick's polar question. (For other uses of the `certain that' diagnostic see \citealt{tonhauser-salt26,stevens-etal2017}).) The `asking whether' response question in (\ref{stim}b) assesses the extent to which Patrick is asking about the projective content, i.e., the extent to which the projective content is at-issue in Patrick's polar question. This diagnostic for at-issueness relies on the assumption that the context set is more likely to be partitioned by at-issue content and its negation, than by not-at-issue content and its negation. For other uses of this diagnostic of at-issueness see, e.g., \citealt{amaral-etal07} and \citealt{tonhauser-sula6}.

\begin{exe}

\ex\label{stim} Patrick asks: {\em Was Martha's new car, a BMW, expensive?} 

\begin{xlist}
\ex `certain that' question (projectivity): Is Patrick certain that Martha's new car is a BMW?

\ex `asking whether' question (at-issueness): Is Patrick asking whether Martha's new car is a BMW?

\end{xlist}

\end{exe}
In Exps.~1a and 1b, each participant was asked to respond to both the `certain that'  and the `asking whether' response questions for each experimental item they were presented with.

\subsection{Experiment 1a}\label{s-exp1a}

Exp.~1a was designed to explore the projectivity and at-issueness of the 9 projective contents occurring with the syntactically heterogeneous target expressions in (\ref{pairs1a2a}), i.e., the content of NRRCs and nominal appositives, the possession implication of possessive noun phrases, the prejacent of {\em only} and {\em stupid}, the pre-state implication of {\em stop} and the contents of the complements of {\em annoyed, discover} and {\em know}.

\subsubsection{Methods}\label{s-methods-1a}

\paragraph{Participants.} 250 participants with U.S.\ IP addresses and at least 97\% of previous HITs approved were recruited on Amazon's Mechanical Turk platform (ages: 19-71; median: 33). They were paid \$1 for participating in the experiment. 

\paragraph{Materials.} The 9 projective contents explored in this experiment were instantiated by 17 lexical contents, which are shown in (\ref{contents}) together with the label used to refer to the lexical content. Recall that we use the term `projective content' to refer to the abstract characterization of the projective content of an expression (e.g., for {\em discover}, the content of the clausal complement) and the term `lexical content' to refer to the lexical content with which the projective content is instantiated (see footnote \ref{f-content}).


\begin{exe}
\ex\label{contents} 17 lexical contents used in Exp.~1a and 2a

\begin{enumerate}[itemsep=-.5mm]

\ex muffins: these muffins have blueberries in them

\ex pizza: this pizza has mushrooms on it

\ex play: Jack was playing outside with the kids

\ex vegetarian: Don is a vegetarian

\ex cheat: Raul cheated on his wife

\ex nails: Mary's daughter has been biting her nails

\ex ballet: Ann used to dance ballet

\ex kids: John's kids were in the garage

\ex hat: Samantha has a new hat

\ex bmw: Martha has a new BMW

\ex boyfriend: Betsy has a boyfriend

\ex alcatraz: Mike visited Alcatraz

\ex aunt: Janet has a sick aunt

\ex cupcakes: Marissa brought the cupcakes

\ex soccer: the soccer ball has a hole in it

\ex olives: this bread has olives in it

\ex stuntman: Richie is a stuntman

\end{enumerate}
\end{exe}

Each of the 9 projective contents was instantiated by 3-8 of the 17 lexical contents, as shown in \tableref{t-trigger-content-pairs}. As discussed in \sectionref{s1}, different projective contents were instantiated by the same lexical contents (e.g., the lexical content `muffins' instantiated the projective content of NRRCs and {\em only}) to test for independent contributions of the target expressions and the lexical content to potential variability in projectivity. As shown in \tableref{t-trigger-content-pairs}, each lexical content instantiated between 2 and 4 projective contents.

\begin{table}[!h]
\begin{center}
\begin{tabular}{l|ccccccccc}
{\bf lexical} & \multicolumn{9}{c}{\bf Target expression} \\ 
 
{\bf content} & NRRC & NomApp & possNP & {\em discover} & {\em know} & {\em annoyed} & {\em stop} & {\em only} & {\em stupid} \\\hline \hline

muffins & $\checkmark$ & & & & & & & $\checkmark$ &  \\

\hline

kids & & & & & & & & $\checkmark$ & $\checkmark$ \\

\hline

pizza & & & & $\checkmark$ & & $\checkmark$ & & $\checkmark$ &  \\

\hline

play & & & & $\checkmark$ & $\checkmark$ & & $\checkmark$ & &  \\

\hline

nails & & & & $\checkmark$ & & & $\checkmark$ & & $\checkmark$  \\

\hline

ballet & & $\checkmark$& & & & & $\checkmark$ & &  \\

\hline

cheat & & & & & $\checkmark$ & & & & $\checkmark$ \\

\hline

stuntman & & $\checkmark$ & & & & & & & $\checkmark$ \\

\hline

bmw & & $\checkmark$ & $\checkmark$ & & $\checkmark$ & $\checkmark$ & & &  \\

\hline

vegetarian & $\checkmark$ & $\checkmark$& & & & & & &  \\

\hline

hat & & & $\checkmark$ & & $\checkmark$ & $\checkmark$ & & &  \\

\hline

boyfriend & $\checkmark$ & & $\checkmark$ & & & & & &  \\

\hline

aunt & $\checkmark$ & & $\checkmark$ & & $\checkmark$ & & & &  \\

\hline

alcatraz & $\checkmark$ & & & $\checkmark$ & $\checkmark$ & & & &  \\

\hline

soccer & $\checkmark$ & & & $\checkmark$ & $\checkmark$ & $\checkmark$ & & &  \\

\hline

olives & & & & & & $\checkmark$ & & &  \\

\hline

cupcakes & $\checkmark$ & & & & $\checkmark$ & & & &  \\

\hline

%only: muffins","kids","pizza"], 3
%"stop":["play","nails","ballet"],	3
%"stupid":["kids","cheat","nails","stuntman"], 4
%"NomApp":["bmw","veggie","ballet","stuntman"], 4
%"possNP":["hat","bmw","boyfriend","aunt"],     4
%"discover":["play","pizza","nails","alcatraz","soccer"], 5
%"annoyed":["hat","bmw","pizza","soccer","olives"], 5
%"NRRC":["veggie","boyfriend","muffins","alcatraz","aunt","cupcakes","soccer"], 7
%"know":["play","cheat","hat","alcatraz","aunt","cupcakes","soccer","bmw"], 8
\end{tabular}
\end{center}
\caption{Lexical contents instantiating the projective contents of the 9 target expressions in Exps.~1a and 2a. Abbreviations: NRRC = non-restrictive relative clause, NomApp = nominal appositive, possNP = possessive noun phrase.}\label{t-trigger-content-pairs}
\end{table}

The target stimuli were polar questions asked by a speaker, like those in (\ref{target}a) and (\ref{target}b), in which the lexical content `stuntman' instantiates the projective contents of a nominal appositive and of {\em stupid}, respectively:

\begin{exe}
\ex\label{target}
\begin{xlist}
\ex Did Richie, a stuntman, break his leg?
\ex Is Richie stupid to be a stuntman?
\end{xlist}
\end{exe}

The experiment also included 17 control stimuli, which were polar questions formed from the 17 lexical contents in (\ref{contents}): for example, the control stimulus formed from the lexical content `stuntman' is shown in (\ref{control}). The control stimuli were included to confront participants with contents that are at-issue and not projective, and to assess whether participants were attending to the task. 

\begin{exe}
\ex\label{control} Is Richie a stuntman?
\end{exe}
The full set of stimuli of Exp.~1a is provided in Appendix \ref{a-exp1a-2a-stimuli}.

Each participant saw a random set of 15 polar questions. Each set contained a target polar question for each of the 9 projective contents (each instantiated by a unique lexical content) and 6 control polar questions (with unique lexical contents as well), for a total of 15 unique lexical contents from (\ref{contents}). Each participant saw their 15 polar questions twice, for a total of 30 trials: in one block, participants responded to `certain that' questions to assess projectivity, as shown in (\ref{stim}a). In the other block, participants responded to `asking whether' questions to assess at-issueness, as shown in (\ref{stim}b). Block order and within-block trial order were randomized.

\paragraph{Procedure.} Participants were told to imagine that they are at a party and that, upon walking into the kitchen, they overhear somebody ask another person a question. On each trial, participants read the polar question produced by a random speaker as well as the corresponding response question, and then gave their response on a slider marked `no' at one end and `yes' at the other, as shown in \figref{f-trial-exp1} for a trial in an `asking whether' at-issueness block.  


\begin{figure}[h!]
\begin{center}
\fbox{\includegraphics[width=12cm]{figures/exp1-trial}}
\end{center}
\caption{A sample (at-issueness) trial in Exps.~1a and 1b}\label{f-trial-exp1}
\end{figure}

A `yes' response to a `certain that' question was taken to indicate that the person who uttered the polar question (e.g., Michelle in the sample trial) was committed to the relevant lexical content, i.e., that the lexical content projects, whereas a `no' response was taken to indicate that the lexical content did not project. For the `asking whether' questions, a `yes'  response was taken to indicate that the speaker was asking about the relevant lexical content, i.e., that it was at-issue, whereas a `no' response was taken to indicate that the lexical content was not at-issue. To explore the hypothesis that projectivity and not-at-issueness are positively related,  `yes' responses to `certain that' questions and `no' responses to `asking whether' questions were coded as 1; accordingly, `no' responses to `certain that' questions and `yes' responses to `asking whether' questions were coded as 0.

After completing the experiment, participants filled out a short optional survey about their age, their native language(s) and, if English is their native language, whether they are a speaker of American English (as opposed to, e.g., Australian or Indian English). To encourage them to respond truthfully, participants were told that they would be paid no matter what answers they gave in the survey.

\paragraph{Data exclusion.}
Prior to analysis, the data from 29 participants who did not self-identify as native speakers of American English were excluded. For the remaining 221 participants, we inspected their response means to the `certain that' and `asking whether' questions 
to the main clause controls: for these stimuli, we expect low responses to both types of questions since main clause contents are expected to be at-issue and not project. The response means of 11 participants were more than 3 standard deviations above the group means for at least one type of question (the group means were 0.07 for `certain that' and 0.04 for `asking whether' questions). Closer inspection revealed that these participants' responses to the control polar questions were systematically higher than the group means and involved 16 of the 17 lexical contents, suggesting that these participants did not attend to the task or interpreted the task differently. The data from these 11 participants were also excluded, leaving data from 210 participants (ages 19-68; median: 33).  


\subsubsection{Results}

We begin by addressing the two main questions of interest, namely whether there is projection variability across the projective contents of the target expressions and whether projectivity a function of at-issueness, as predicted by the Projection Principle. 

\paragraph{Projection variability across projective contents.} By-projective content variability can be seen in \figref{fig:proj-triggmeans}:  while median projectivity ratings were all close to ceiling (suggesting that for each projective content at least half of the participants took it to be robustly projective), the variable mean responses, box sizes and whisker lengths provide evidence of variability in projectivity across target expressions. For example, the mean projectivity of the prejacent of \emph{only} was relatively low at .76, while the mean projectivity of the projective contents of NRRCs and \emph{annoyed} was close to ceiling at .96. \figref{fig:proj-subjmeans} shows that about one third of participants took the 9 projective contents they judged to be robustly projective. For the remaining participants, the decreasing means (from right to left) reveal a decrease in the overall projectivity of the 9 projective contents and the increasingly larger error bars reveal an increase in projection variability among the 9 projective contents. In sum, there is projection variability across the 9 projective contents.

\begin{figure}[h!]
\centering

\subfloat[][Boxplot of projection variability by expression, including main clause controls and collapsing across lexical contents. Blue dots indicate trigger means and notches indicate medians. `MC' abbreviates main clause.]{ 
	\includegraphics[width=12cm]{../results/exp1a/graphs/boxplot-projection-with-MCs}
	\label{fig:proj-triggmeans}
}

\subfloat[][Projectivity means by participant (excluding main clause controls). Error bars indicate bootstrapped 95\% confidence intervals.]{ 
	\includegraphics[width=12cm]{../results/exp1a/graphs/projection-subjectmeans}
	\label{fig:proj-subjmeans}
	}
	

\caption{Projectivity by expression (top panel) and by participant (bottom panel)}
\label{fig:f-proj-1a}
\end{figure}

To determine which projective contents differed from each other in projectivity, we conducted post hoc pairwise comparisons using Tukey's method (allowing for by-participant variability), using the \verb|lsmeans| package \citep{tukey} in R \citep{r}. P-values for each pair of target expression/projective content are displayed in \tableref{tab:pairwise}. These results suggest no difference in the projectivity of the projective contents of NRRCs, \emph{annoyed}, nominal appositives, possessive NPs, and \emph{know}. The projective contents of the other target expressions differed from each other in projectivity, except for the pairs \emph{discover/know} (which displayed only a marginally significant difference), \emph{discover/stop}, \emph{stupid/discover}, and \emph{stupid/stop}. 

%$lsmeans
% short_trigger    lsmean         SE      df  lower.CL  upper.CL
% annoyed       0.9595714 0.01473729 1675.74 0.9306660 0.9884769
% discover      0.8556667 0.01473729 1675.74 0.8267612 0.8845721
% know          0.9233810 0.01473729 1675.74 0.8944755 0.9522864
% NomApp        0.9528571 0.01473729 1675.74 0.9239517 0.9817626
% NRRC          0.9637619 0.01473729 1675.74 0.9348565 0.9926673
% only          0.7604286 0.01473729 1675.74 0.7315231 0.7893340
% possNP        0.9386667 0.01473729 1675.74 0.9097612 0.9675721
% stop          0.8650476 0.01473729 1675.74 0.8361422 0.8939531
% stupid        0.8455238 0.01473729 1675.74 0.8166184 0.8744293
%
%Degrees-of-freedom method: satterthwaite 
%Confidence level used: 0.95 
%
%$contrasts
% contrast               estimate         SE   df t.ratio p.value
% discover - annoyed -0.103904762 0.01947976 1680  -5.334  <.0001
% know - annoyed     -0.036190476 0.01947976 1680  -1.858  0.6431
% know - discover     0.067714286 0.01947976 1680   3.476  0.0153
% NomApp - annoyed   -0.006714286 0.01947976 1680  -0.345  1.0000
% NomApp - discover   0.097190476 0.01947976 1680   4.989  <.0001
% NomApp - know       0.029476190 0.01947976 1680   1.513  0.8496
% NRRC - annoyed      0.004190476 0.01947976 1680   0.215  1.0000
% NRRC - discover     0.108095238 0.01947976 1680   5.549  <.0001
% NRRC - know         0.040380952 0.01947976 1680   2.073  0.4923
% NRRC - NomApp       0.010904762 0.01947976 1680   0.560  0.9998
% only - annoyed     -0.199142857 0.01947976 1680 -10.223  <.0001
% only - discover    -0.095238095 0.01947976 1680  -4.889  <.0001
% only - know        -0.162952381 0.01947976 1680  -8.365  <.0001
% only - NomApp      -0.192428571 0.01947976 1680  -9.878  <.0001
% only - NRRC        -0.203333333 0.01947976 1680 -10.438  <.0001
% possNP - annoyed   -0.020904762 0.01947976 1680  -1.073  0.9780
% possNP - discover   0.083000000 0.01947976 1680   4.261  0.0007
% possNP - know       0.015285714 0.01947976 1680   0.785  0.9973
% possNP - NomApp    -0.014190476 0.01947976 1680  -0.728  0.9984
% possNP - NRRC      -0.025095238 0.01947976 1680  -1.288  0.9349
% possNP - only       0.178238095 0.01947976 1680   9.150  <.0001
% stop - annoyed     -0.094523810 0.01947976 1680  -4.852  <.0001
% stop - discover     0.009380952 0.01947976 1680   0.482  0.9999
% stop - know        -0.058333333 0.01947976 1680  -2.995  0.0689
% stop - NomApp      -0.087809524 0.01947976 1680  -4.508  0.0002
% stop - NRRC        -0.098714286 0.01947976 1680  -5.068  <.0001
% stop - only         0.104619048 0.01947976 1680   5.371  <.0001
% stop - possNP      -0.073619048 0.01947976 1680  -3.779  0.0051
% stupid - annoyed   -0.114047619 0.01947976 1680  -5.855  <.0001
% stupid - discover  -0.010142857 0.01947976 1680  -0.521  0.9999
% stupid - know      -0.077857143 0.01947976 1680  -3.997  0.0022
% stupid - NomApp    -0.107333333 0.01947976 1680  -5.510  <.0001
% stupid - NRRC      -0.118238095 0.01947976 1680  -6.070  <.0001
% stupid - only       0.085095238 0.01947976 1680   4.368  0.0005
% stupid - possNP    -0.093142857 0.01947976 1680  -4.782  0.0001
% stupid - stop      -0.019523810 0.01947976 1680  -1.002  0.9857
%
%P value adjustment: tukey method for comparing a family of 9 estimates  

\begin{table}[!h]
\begin{center}
\begin{tabular}{r c c c c c c c c}
\toprule
 &   NRRC & annoyed & NomApp &  possNP &  know & stop & discover & stupid \\
 \midrule
annoyed &  \emph{ns}  &  -  &        -   &       -  &        -  &   - &     -   &  -\\     
NomApp  &  \emph{ns} & \emph{ns} & -    &   -   &    -    &   -  &     -   & - \\    
possNP  &    \emph{ns} & \emph{ns} & \emph{ns} & - &      -  &     -     &  -   & - \\    
know     &   \emph{ns} & \emph{ns} & \emph{ns} & \emph{ns} & -   &    -   &    -       & -\\
stop     &   *** & *** & ** & ** & . & - &  - & -\\      
discover  &   *** & *** & *** & ** & * & \emph{ns} & - & -      \\
stupid    &  *** & *** & *** & *** & ** & \emph{ns} & \emph{ns} & - \\
only      &  *** & *** & *** & *** & *** & *** & *** & ** \\
\bottomrule
\end{tabular}
\caption{P-values associated with pairwise comparison of projective content projectivity means using Tukey's method. `***' indicates significance at .0001, `**' at .01, `*' at .05, `.' marginal significance at .1, and \emph{n.s} indicates no significant difference in means.}\label{tab:pairwise}
\end{center}
\end{table}

This brings us to our second research question: is projectivity a function of at-issueness, as predicted by the Projection Principle?

\paragraph{At-issueness.} Mean projectivity ratings for each target expression are visualized as a function of their mean not-at-issueness ratings in \figref{fig:f-proj-ai-1a}. There is a clear relationship between at-issueness and projectivity: projective contents that received higher projectivity ratings were also considered to be more not-at-issue.

\begin{figure}[!h]

\begin{center}
%\includegraphics[width=12cm]{../results/exp1a/graphs/ai-proj-bytrigger}
\includegraphics[width=10cm]{../results/exp1a/graphs/ai-proj-bytrigger-labels}
\end{center}

\caption{Mean projectivity against mean not-at-issueness by target expression/projective content. Error bars indicate bootstrapped 95\% confidence intervals. Dashed line indicates perfect correlation line.}
\label{fig:f-proj-ai-1a}
\end{figure}

This qualitative observation about the relation between at-issueness and projectivity was borne out statistically. We conducted a mixed-effects linear regression predicting projectivity rating from a centered fixed effect of at-issueness rating. In order to control for block order effects, the model also included a centered fixed effect of block order and the interaction of block order and at-issueness. The model included the maximal random effects structure justified by the data and the theoretical questions: random by-expression intercepts (capturing differences in projectivity between target expressions),  random by-lexical content intercepts (capturing differences in projectivity between lexical contents), random by-participant intercepts (capturing individual variability in projectivity) and random slopes for at-issueness by target expression, lexical content, and participant (capturing that the effect of at-issueness may vary across target expressions, lexical contents, and participants). Here and in the remainder of the paper, p-values were obtained by likelihood ratio tests of the full model with the effect in question against the model without the effect in question. The analysis was conducted on non-main-clause trials only (1890 data points) because we were specifically interested in variability in projectivity for contents that have the potential to project. Analyses were conducted using the \verb|lme4| package \citep{bates2015}.

There was a significant main effect of at-issueness such that more not-at-issue items received higher projectivity ratings ($\beta$ = 0.37, $SE$ = 0.10, $t$ = 3.70, $\chi^2(1)$ = 9.20, $p <$ .003). This suggests that projectivity a function of at-issueness, i.e., the information-structural status of projective content, as predicted by the Projection Principle. Likelihood ratio tests revealed that each random effect was justified (see \tableref{tab:random1a} for standard deviations and p-values); that is, there was by-participant, by-expression and by-lexical content variability in projectivity, as well as variability in the at-issueness effect across participants, target expressions and lexical contents. This finding suggests that there are target expression-specific, conventional, effects in projectivity. The block effect did not reach significance ($\beta$ = -0.02, $SE$ = 0.01, $t$ = -1.56, $\chi^2(1)$ = 2.39, $p >$ .12), nor did the interaction term ($\beta$ = 0.08, $SE$ = 0.08, $t$ = 0.98, $\chi^2(1)$ = 0.96, $p >$ .32), suggesting that the order in which participants completed the tasks (projectivity, at-issueness) did not affect their judgments in a systematic way. 

% Random effects:
% Groups        Name        Variance Std.Dev. Corr 
% workerid      (Intercept) 0.003534 0.05945       
%               cai         0.152839 0.39095  -0.69
% content       (Intercept) 0.000613 0.02476       
%               cai         0.034367 0.18538  -0.74
% short_trigger (Intercept) 0.001353 0.03678       
%               cai         0.048460 0.22014  -0.73
% Residual                  0.025980 0.16118       
%Number of obs: 1890, groups:

\begin{table}
\begin{tabular}{c c c c c c }
\toprule
\multicolumn{3}{c}{Intercepts} & \multicolumn{3}{c}{Slopes for at-issueness}\\
Target expression & Lexical content & Participant & Target expression & Lexical content & Participant\\
\midrule
.04 & .02 & .06 & .22 & .19 & .39\\
$< .0001$ & $< .003$ & $< .0001$ & $< .0001$ & $< .0001$ & $< .0001$ \\
\bottomrule
\end{tabular}
\caption{Standard deviations (first row) and p-values (second row, $\chi^2(2)$) for random effects in Exp.~1a model.}\label{tab:random1a}
\end{table}

%
%\begin{figure}[!h]
%\begin{center}
%
%\includegraphics[width=12cm]{../results/exp1a/graphs/ai-subjectmeans}
%
%\includegraphics[width=12cm]{../results/exp1a/graphs/boxplot-not-at-issueness}
%
%\end{center}
%\caption{At-issueness by participant (top panel) and projective content trigger (bottom panel) \jd{not sure this is an interesting plot}}
%\label{f-ai-1a}
%\end{figure}
%
%


\subsubsection{Discussion}\label{s-discussion1a}

Exp.~1a was designed to explore projection variability for projective contents associated with a set of syntactically heterogeneous target expressions and to test the Projection Principle, which holds that the projectivity of projective content is related to its information-structural status. The experiment provided robust empirical evidence for projection variability across the 9 projective contents. Furthermore, the experiment provided evidence for projection variability across participants. This finding suggests that speakers of American English differ in the extent to which they take the projective contents associated with the 9 target expressions to project. Finally, the experiment also showed that the lexical content that instantiates a projective content plays a role in the extent to which the projective content projects. A methodological implication of these latter two findings is that research on projective content must be sensitive to potential inter-speaker and inter-lexical content differences.

The experiment also provided empirical support for the Projection Principle since the at-issueness of projective content was a significant predictor of its projectivity. This finding suggests that the extent to which a speaker is taken to be committed to a particular projective content is related to the extent to which the projective content is at-issue in the speaker's utterance. The experiment also showed that the target expressions differ in the extent to which the projective content they are associated with projects. This finding suggests that both the information-structural status of the projective content and the conventional meaning of the target expressions may play a role the extent to which a speaker is taken to be committed to a projective content. As mentioned in section \ref{s1}, we discuss the implications of this finding in section \ref{s5}.

\subsection{Experiment 1b}\label{s-exp1b} 

Exp.~1b was designed to explore the projectivity and at-issueness of the projective contents of the 12 syntactically more homogeneous predicates in (\ref{pairs1b2b}), i.e., the contents of the complements of the emotive factives {\em be amused} and {\em be annoyed}, the cognitive factives {\em know} and {\em be aware}, the sensory factive {\em see}, the cognitive semi-factives {\em discover, find out, realize, learn} and {\em establish}, and the communication semi-factives {\em confess} and {\em reveal}.


\subsubsection{Methods}

\paragraph{Participants.} 250 participants with U.S.\ IP addresses and at least 97\% of previous HITs approved were recruited on Amazon's Mechanical Turk platform (ages: 18-74; median: 32). They were paid \$1 for participating in the experiment.

\paragraph{Materials.} The 12 projective contents explored in this experiment were instantiated by the 20 lexical contents shown in (\ref{contents2}). Each of the 12 projective contents was instantiated by each of these 20 lexical contents, for a total of 240 target stimuli. 

\begin{exe}
\ex\label{contents2} 20 lexical contents in Exps.~1b and 2b

\begin{enumerate}[itemsep=-.5mm]

\begin{multicols}{2}
\item Raul was drinking chamomile tea
\item Jack played frisbee with the kids
\item John was hiding in the garage
\item Mike visited the zoo
\item Zach dyed his hair purple
\item Marissa brought almond cupcakes
\item Chad put up a swing in his backyard
\item Greg drove his car into a ditch
\item Kate fell from her horse
\item Joyce got a poodle 
\columnbreak
\item Carl wrote a poem for his wife
\item Bea posted a family picture on Facebook
\item Janet moved into a damp apartment
\item Samantha bought a fur hat
\item Don ate a chili dog
\item Mary was biting her nails
\item Richie jumped into the pool
\item Martha came in her new BMW
\item Ann was dancing in the corner
\item Sue was doing yoga in the yard
\end{multicols}
\end{enumerate}

\end{exe}

The target stimuli were (past and present tense)\footnote{The main clauses of stimuli with {\em be amused, be aware} and {\em be annoyed} were realized in the present tense; the main clauses of stimuli with the other predicates were realized in the past tense.} polar questions formed from sentences with one of the 12 predicates, a clausal complement formed from one of the 20 lexical contents in  (\ref{contents2}) and a random proper name subject (the names used for the subjects did not occur in the clausal complements or as the speakers). Two sample target stimuli are given in (\ref{sample2}): the complement clause in both stimuli is instantiated by the lexical content 1.\ in (\ref{contents2}), namely `Raul was drinking chamomile tea'.

\begin{exe}
\ex\label{sample2}
\begin{xlist}
\ex Is Shirley aware that Raul was drinking chamomile tea?

\ex Did Samuel discover that Raul was drinking chamomile tea?
\end{xlist}
\end{exe}

The experiment also included 20 control stimuli, which were (past tense) polar questions formed from sentences conveying the 20 lexical contents in (\ref{contents2}); a sample control polar question formed from the lexical content 1.\ in (\ref{contents2}) is shown in (\ref{sample3}). The control stimuli were included to confront participants with contents that are at-issue and not projective, and to assess whether participants were attending to the task.

\begin{exe}
\ex\label{sample3} Was Raul drinking chamomile tea?
\end{exe}

For each participant, a set of 20 polar question stimuli was randomly created: each set contained a target polar question for each of the 12 target expressions (the projective content of each expression was instantiated by a unique lexical content) and 8 control polar questions (with unique lexical contents as well, for a total of 20 unique lexical contents from (\ref{contents2})). Each participants saw their 20 polar question stimuli twice, for a total of 40 trials: in one block, participants responded to `certain that' questions to assess projectivity; in the other block, participants responded to `asking whether' questions to assess at-issueness. Block order and within-block trial order were randomized.

\paragraph{Procedure.} The procedure of Exp.~1b was the same as for Exp.~1a, described in section \ref{s-methods-1a}, except that participants responded to 20 `certain that' and 20 `asking whether' questions. (There are more trials in Exp.~1b than in Exp.~1a because each participant judged the projectivity and at-issueness of 20  rather than 15 contents.)


\paragraph{Data exclusion.} Prior to analysis, we excluded the data from 3 participants who did not self-identify as native speakers of American English. For the remaining 247 participants, we inspected their response means to the `certain that' and `asking whether' questions to the main clause controls. The response means of 12 participants were more than 3 standard deviations above the group means for at least one type of question (the group means were 0.08 for `certain that' and 0.04 for `asking whether' questions). Further inspection revealed that these participants' responses to the control questions were systematically higher than the group means and involved all of the 20 lexical contents, suggesting that these participants did not attend to the task or interpreted the task differently. The data from these 12 participants were also excluded, leaving data from 235 participants (ages 18-74; median: 33).

\subsubsection{Results}

\paragraph{Projection variability across projective contents.} We begin by addressing the research question in (\ref{questions}a), whether the projective contents associated with the 12 target expressions exhibit projection variability. Variability in how robustly the contents of the clausal complements of the 12 (semi-)factive predicates project can be seen in \figref{fig:proj-triggmeans-1b}. The median projectivity ratings are at ceiling for the contents of the clausal complements of most of the (semi-)factive predicates, suggesting that at least half of the participants took these contents to be robustly projective. The exceptions are the predicates {\em established, confessed} and {\em revealed}, for which the medians are not at ceiling, suggesting that the speaker was less likely to be taken to be committed to the contents of the complements of these predicates than for the other 9 predicates. Variability in how robustly the contents of the clausal complements of the 12 predicates is also suggested by variable mean responses, box sizes and whisker lengths across the projective contents of the 12 predicates. \figref{fig:proj-subjmeans-1b} shows that relatively few participants took the contents of the complements of all 12 predicates to be robustly projective. Instead, the mean responses suggest that there is by-participant variability in how robustly the contents of the complements of the 12 predicates were taken to project.

\begin{figure}[!h]
\centering

\subfloat[][Boxplot of projection variability by expression, including main clause controls and collapsing across lexical contents. Blue dots indicate trigger means and notches indicate medians. `MC' abbreviates main clause.]{ 
	\includegraphics[width=16cm]{../results/exp1b/graphs/boxplot-projection-with-MCs}
	\label{fig:proj-triggmeans-1b}
}

\subfloat[][Projectivity means by participant (excluding main clause controls). Error bars indicate bootstrapped 95\% confidence intervals.]{ 
	\includegraphics[width=16cm]{../results/exp1b/graphs/projection-subjectmeans}
	\label{fig:proj-subjmeans-1b}
	}
	
\caption{Projectivity by expression (top panel) and by participant (bottom panel)}\label{fig:f-proj-1b}
\end{figure}

As in Exp.~1a,  we conducted post hoc pairwise comparisons using Tukey's method to determine which of the projective contents associated with the 12 target expressions differed from each other in projectivity. P-values for each pair of target expression/projective content are displayed in \tableref{tab:pairwise-1b}. The results suggest no difference in the projectivity of the projective contents of \emph{annoyed}, \emph{noticed}, \emph{aware}, \emph{realize}, \emph{amused}, \emph{saw}, and \emph{found out}. The projective contents of the other predicates differed from each other in projectivity, with the exception of the pairs \emph{discover/saw}, \emph{discover/found out}, and \emph{discover/learned}. Furthermore, \emph{discover} was  marginally different from \emph{aware}, \emph{realize}, and \emph{amused}).

%$lsmeans
% short_trigger    lsmean         SE     df  lower.CL  upper.CL
% confessed     0.6869787 0.01499147 1739.4 0.6575755 0.7163819
% discovered    0.8534043 0.01499147 1739.4 0.8240011 0.8828075
% established   0.4177447 0.01499147 1739.4 0.3883415 0.4471479
% found_out     0.8837447 0.01499147 1739.4 0.8543415 0.9131479
% is_amused     0.9102979 0.01499147 1739.4 0.8808947 0.9397011
% is_annoyed    0.9228936 0.01499147 1739.4 0.8934904 0.9522968
% is_aware      0.9208936 0.01499147 1739.4 0.8914904 0.9502968
% learned       0.8821277 0.01499147 1739.4 0.8527245 0.9115309
% noticed       0.9212340 0.01499147 1739.4 0.8918308 0.9506372
% realize       0.9133191 0.01499147 1739.4 0.8839159 0.9427224
% revealed      0.7784255 0.01499147 1739.4 0.7490223 0.8078287
% saw           0.8905957 0.01499147 1739.4 0.8611925 0.9199989
%
%Degrees-of-freedom method: satterthwaite 
%Confidence level used: 0.95 
%
%$contrasts
% contrast                      estimate         SE   df t.ratio p.value
% discovered - confessed    0.1664255319 0.01851128 2585   8.990  <.0001
% established - confessed  -0.2692340426 0.01851128 2585 -14.544  <.0001
% established - discovered -0.4356595745 0.01851128 2585 -23.535  <.0001
% found_out - confessed     0.1967659574 0.01851128 2585  10.630  <.0001
% found_out - discovered    0.0303404255 0.01851128 2585   1.639  0.8946
% found_out - established   0.4660000000 0.01851128 2585  25.174  <.0001
% is_amused - confessed     0.2233191489 0.01851128 2585  12.064  <.0001
% is_amused - discovered    0.0568936170 0.01851128 2585   3.073  0.0892
% is_amused - established   0.4925531915 0.01851128 2585  26.608  <.0001
% is_amused - found_out     0.0265531915 0.01851128 2585   1.434  0.9568
% is_annoyed - confessed    0.2359148936 0.01851128 2585  12.744  <.0001
% is_annoyed - discovered   0.0694893617 0.01851128 2585   3.754  0.0097
% is_annoyed - established  0.5051489362 0.01851128 2585  27.289  <.0001
% is_annoyed - found_out    0.0391489362 0.01851128 2585   2.115  0.6121
% is_annoyed - is_amused    0.0125957447 0.01851128 2585   0.680  0.9999
% is_aware - confessed      0.2339148936 0.01851128 2585  12.636  <.0001
% is_aware - discovered     0.0674893617 0.01851128 2585   3.646  0.0144
% is_aware - established    0.5031489362 0.01851128 2585  27.181  <.0001
% is_aware - found_out      0.0371489362 0.01851128 2585   2.007  0.6889
% is_aware - is_amused      0.0105957447 0.01851128 2585   0.572  1.0000
% is_aware - is_annoyed    -0.0020000000 0.01851128 2585  -0.108  1.0000
% learned - confessed       0.1951489362 0.01851128 2585  10.542  <.0001
% learned - discovered      0.0287234043 0.01851128 2585   1.552  0.9258
% learned - established     0.4643829787 0.01851128 2585  25.086  <.0001
% learned - found_out      -0.0016170213 0.01851128 2585  -0.087  1.0000
% learned - is_amused      -0.0281702128 0.01851128 2585  -1.522  0.9348
% learned - is_annoyed     -0.0407659574 0.01851128 2585  -2.202  0.5481
% learned - is_aware       -0.0387659574 0.01851128 2585  -2.094  0.6271
% noticed - confessed       0.2342553191 0.01851128 2585  12.655  <.0001
% noticed - discovered      0.0678297872 0.01851128 2585   3.664  0.0135
% noticed - established     0.5034893617 0.01851128 2585  27.199  <.0001
% noticed - found_out       0.0374893617 0.01851128 2585   2.025  0.6761
% noticed - is_amused       0.0109361702 0.01851128 2585   0.591  1.0000
% noticed - is_annoyed     -0.0016595745 0.01851128 2585  -0.090  1.0000
% noticed - is_aware        0.0003404255 0.01851128 2585   0.018  1.0000
% noticed - learned         0.0391063830 0.01851128 2585   2.113  0.6138
% realize - confessed       0.2263404255 0.01851128 2585  12.227  <.0001
% realize - discovered      0.0599148936 0.01851128 2585   3.237  0.0555
% realize - established     0.4955744681 0.01851128 2585  26.771  <.0001
% realize - found_out       0.0295744681 0.01851128 2585   1.598  0.9102
% realize - is_amused       0.0030212766 0.01851128 2585   0.163  1.0000
% realize - is_annoyed     -0.0095744681 0.01851128 2585  -0.517  1.0000
% realize - is_aware       -0.0075744681 0.01851128 2585  -0.409  1.0000
% realize - learned         0.0311914894 0.01851128 2585   1.685  0.8753
% realize - noticed        -0.0079148936 0.01851128 2585  -0.428  1.0000
% revealed - confessed      0.0914468085 0.01851128 2585   4.940  0.0001
% revealed - discovered    -0.0749787234 0.01851128 2585  -4.050  0.0031
% revealed - established    0.3606808511 0.01851128 2585  19.484  <.0001
% revealed - found_out     -0.1053191489 0.01851128 2585  -5.689  <.0001
% revealed - is_amused     -0.1318723404 0.01851128 2585  -7.124  <.0001
% revealed - is_annoyed    -0.1444680851 0.01851128 2585  -7.804  <.0001
% revealed - is_aware      -0.1424680851 0.01851128 2585  -7.696  <.0001
% revealed - learned       -0.1037021277 0.01851128 2585  -5.602  <.0001
% revealed - noticed       -0.1428085106 0.01851128 2585  -7.715  <.0001
% revealed - realize       -0.1348936170 0.01851128 2585  -7.287  <.0001
% saw - confessed           0.2036170213 0.01851128 2585  11.000  <.0001
% saw - discovered          0.0371914894 0.01851128 2585   2.009  0.6873
% saw - established         0.4728510638 0.01851128 2585  25.544  <.0001
% saw - found_out           0.0068510638 0.01851128 2585   0.370  1.0000
% saw - is_amused          -0.0197021277 0.01851128 2585  -1.064  0.9960
% saw - is_annoyed         -0.0322978723 0.01851128 2585  -1.745  0.8472
% saw - is_aware           -0.0302978723 0.01851128 2585  -1.637  0.8955
% saw - learned             0.0084680851 0.01851128 2585   0.457  1.0000
% saw - noticed            -0.0306382979 0.01851128 2585  -1.655  0.8881
% saw - realize            -0.0227234043 0.01851128 2585  -1.228  0.9868
% saw - revealed            0.1121702128 0.01851128 2585   6.060  <.0001
%
%P value adjustment: tukey method for comparing a family of 12 estimates


\begin{table}[!h]
\begin{center}
\begin{tabular}{l l l l l l l l l l l l }
\toprule
 &   \rot{annoyed} & \rot{noticed} & \rot{aware} &  \rot{realize} &  \rot{amused} & \rot{saw} & \rot{found out} & \rot{learned} & \rot{discover} & \rot{revealed} & \rot{confessed} \\
 \midrule
noticed & \emph{n.s} & - & - & - & - & - & - & - & - & - & - \\ 
aware & \emph{n.s} & \emph{n.s} & -&- &- &- &- &- &- &- &-  \\ 
realize & \emph{n.s} & \emph{n.s} & \emph{n.s} & - & - &- &- &- &- &- &-  \\ 
amused & \emph{n.s} & \emph{n.s} & \emph{n.s} & \emph{n.s} & - & - & - & - & - & - & -  \\ 
saw & \emph{n.s} & \emph{n.s} & \emph{n.s} & \emph{n.s} & \emph{n.s} & - &- &- &- &- &-  \\ 
found out & \emph{n.s} & \emph{n.s} & \emph{n.s} & \emph{n.s} & \emph{n.s}& \emph{n.s} & -& -& -& -& -\\ 
learned  & \emph{n.s} & \emph{n.s} & \emph{n.s} & \emph{n.s} & \emph{n.s} & \emph{n.s} & \emph{n.s}  & - & - & - & - \\ 
discover & ** & * & . & . & . & \emph{n.s} & \emph{n.s} & \emph{n.s.} & - &- &-  \\ 
revealed  & *** & ***  & *** & *** & *** & *** & *** & *** & ** & -&-  \\   
confessed  & *** & *** & *** & *** & *** & *** & *** &  *** & *** & *** & - \\ 
established    & *** & *** & *** & *** & *** & *** & *** & *** & *** & *** & *** \\
\bottomrule
\end{tabular}

\caption{P-values associated with pairwise comparison of projective contents associated with target expressions using Tukey's method. `***' indicates significance at .0001, `**' at .01, `*' at .05, `.' marginal significance at .1, and \emph{n.s} indicates no significant difference in means.}\label{tab:pairwise-1b}
\end{center}

\end{table}


As for the observed variability in Exp.~1a, we now ask: is projectivity a function of at-issueness, as predicted by the Projection Principle?

\paragraph{At-issueness.} Mean projectivity ratings for the projective content associated with each target expression are visualized in \figref{fig:f-proj-ai-1b} as a function of their mean not-at-issueness ratings. There is a clear relationship between at-issueness and projectivity: projective contents that received higher projectivity ratings were considered to be more not-at-issue. 

\begin{figure}[!h]

\begin{center}
%\includegraphics[width=12cm]{../results/exp1b/graphs/ai-proj-bytrigger}
\includegraphics[width=10cm]{../results/exp1b/graphs/ai-proj-bytrigger-labels}
\end{center}

\caption{Mean projectivity against mean not-at-issueness by target expression. Error bars indicate bootstrapped 95\% confidence intervals. Dashed line indicates perfect correlation line.}
\label{fig:f-proj-ai-1b}
\end{figure}

This qualitative observation about the relation between at-issueness and projectivity was again borne out statistically. We conducted almost exactly the same mixed-effects linear regression analysis as we did for Exp.~1a. The model predicted projectivity rating from centered fixed effects of at-issueness rating, block order, and the interaction of block order and at-issueness. The model included the maximal random effects structure that allowed the model to converge: random by-expression intercepts (capturing differences in projectivity between target expressions),  random by-participant intercepts (capturing individual variability in projectivity), and random slopes for at-issueness by target expression, lexical content, and participant (capturing that the effect of at-issueness may vary across target expressions, lexical contents, and participants). The only difference in model structure to that reported in the previous section is that the current model did not contain random by-lexical content intercepts because there was no by-lexical content intercept variability. As before, the analysis was conducted on non-main-clause trials only (2820 data points).

We observed a significant main effect of at-issueness, such that more not-at-issue items received higher projectivity ratings ($\beta$ = 0.34, $SE$ = 0.04, $t$ = 9.31, $\chi^2(1)$ = 31.36, $p <$ .0001). This suggests again that the information-structural status of a projective content is related to its projectivity, as predicted by the Projection Principle. Likelihood ratio tests revealed that of the included random effects, by-lexical content and by-trigger slopes for at-issueness were not justified (see \tableref{tab:random1a} for standard deviations and p-values); that is, there was by-participant and by-expression variability in projectivity, as well as variability in the at-issueness effect across participants, but in contrast to the data collected in Exp.~1a, there was no variability in the at-issueness effect across target expressions and lexical contents. Together, these findings suggests again that there are target expression-specific, conventional, effects in projectivity, but overall less random variability, especially across lexical contents. 

The block effect did not reach significance ($\beta$ = -0.02, $SE$ = 0.01, $t$ = -1.43, $\chi^2(1)$ = 2.02, $p >$ .15), but the interaction term did ($\beta$ = 0.21, $SE$ = 0.05, $t$ = 3.83, $\chi^2(1)$ = 14.09, $p <$ .0002). Simple effects analysis revealed that this was due to a difference in slope: while there was an effect of at-issueness on projectivity in the predicted direction regardless of block order, the effect was greater in the group of participants who performed the projectivity task first ($\beta$ = 0.44, $SE$ = 0.05, $t$ = 9.33) than in the group who performed the at-issueness task first ($\beta$ = 0.24, $SE$ = 0.04, $t$ = 5.46).

%Random effects:
% Groups        Name        Variance Std.Dev. Corr 
% workerid      (Intercept) 0.008759 0.09359       
%               cai         0.054466 0.23338  -0.29
% content       cai         0.001486 0.03855       
% short_trigger (Intercept) 0.013314 0.11539       
%               cai         0.003151 0.05613  0.30 
% Residual                  0.035808 0.18923       
%Number of obs: 2820, groups:  

\begin{table}
\begin{center}
\begin{tabular}{c c c c c c }
\toprule
\multicolumn{2}{c}{Intercepts} & \multicolumn{3}{c}{Slopes for at-issueness}\\
Target expression & Participant & Target expression & Lexical content & Participant\\
\midrule
.12 & .09 & .06 & .04 & .23\\
$< .0001$ & $< .0001$ & $> .20$ & $> .50$ & $< .0001$ \\
\bottomrule
\end{tabular}
\caption{Standard deviations (first row) and p-values (second row, $\chi^2(2)$) for random effects in Exp.~1b model.}\label{tab:random1b}
\end{center}
\end{table}

%
%\begin{figure}[!h]
%\begin{center}
%
%\includegraphics[width=16cm]{../results/exp1b/graphs/ai-subjectmeans}
%
%\includegraphics[width=16cm]{../results/exp1b/graphs/boxplot-not-at-issueness}
%
%\end{center}
%\caption{At-issueness by participant (top panel) and projective content trigger (bottom panel)}
%\label{f-ai-1b}
%\end{figure}


\subsubsection{Discussion}

Exp.~1b was designed to explore projection variability of contents of the complements of 12 (semi-)factive predicates and to further test the Projection Principle. Exp.~1b replicated the key findings from Exp.~1a: there is evidence for variability in the extent to which projective content projects and for a clear role for at-issueness in projectivity.

Like Exp.~1a, Exp.~1b revealed both by-expression and by-participant projection variability, but there was no evidence in Exp.~1b of by-lexical content variability. This difference between the experiments may be due to a number of factors. First, different lexical contents were included in the two experiments and it is possible that the lexical contents in Exp.~1a were more heterogeneous than those in Exp.~1b. Second, the lexical contents uniformly instantiated the contents of clausal complements in Exp.~1b, but a quite varied set of projective contents in Exp.~1a, which may contribute to the variability observed between the lexical contents. Third, recall that in Exp.~1b, each lexical content was paired with every target expression, whereas in Exp.~1a, each lexical content was paired with only a subset of target expressions. As a result, there were 3-21 data points per lexical content/expression combination in Exp.~1b, but 12-78 data points per lexical content/expression combination in Exp.~1a. This difference in number of ratings obtained may have contributed to the observed difference between the two experiments in by-lexical content variability. The role of the lexical contents in projectivity merits further investigation, which we leave to future research. 

The results of Exp.~1a and 1b also differed in the effects of block order on ratings: block order mattered in Exp.~1b, but not in Exp.~1a. The intricate ways in which task demands affect subsequent tasks also deserve further investigation, which we leave to future research.

A final comparison of the two experiments concerns the ratings obtained for the predicates {\em discover} and {\em annoyed}, which were included in both experiments. Taking into account confidence intervals, the projectivity and at-issueness ratings for each predicate were indistinguishable in Exps.~1a and 1b: the projectivity means were .86 and .85 for {\em discover} for .96 and .92 for \emph{annoyed}, respectively; the at-issueness means were .87 and .89 for \emph{discover} and .97 and .94 for \emph{annoyed}, respectively. Furthermore, the small but significant difference in the projectivity of the contents of the clausal complements of the two predicates was maintained across the two experiments. This observation suggests that participants' responses were not substantially influenced by the other items they encountered, and that the `certain that' and `asking whether' diagnostics are stable methods for estimating projectivity and at-issueness.

\subsection{Summary and discussion of Experiments 1a and 1b}\label{s-summary1a1b}

Exps.~1a and 1b were designed to explore projection variability and the relation between projection and at-issueness for projective contents associated with 19 English expressions, as per the two research questions in (\ref{questions}). Regarding the first research question, the two experiments provided robust empirical evidence for variability across the 19 projective contents. In Exp.~1a, the projective contents associated with NRRCs, nominal appositives, {\em annoyed}, possessive noun phrases and {\em know} were robustly projective and indistinguishable from one another in their projectivity. The projective contents associated with {\em stop, discover} and {\em stupid} were significantly less projective than the aforementioned projective contents (marginally so for the {\em stop/know} pair). And the prejacent of {\em only} was significantly less projective than all other projective contents. Since the target expressions included in Exp.~1a did not include any so-called `hard triggers' and included not only complement-taking predicates, the observed differences do not reflect the distinction between `hard' and `soft' triggers, or between `factive' and `semi-factive' predicates (see section \ref{s1} for discussion and references). Rather, the projective contents explored in Exp.~1a exhibit a pattern of projection variability that has not been previously described. Implications for analyses of projection section are discussed in \ref{s5}.

The findings of Exps.~1a and 1b provide empirical support for contents of clausal complements of complement-taking predicates varying in projectivity, and the observed variation partially aligns with the distinction between `factive' and `semi-factive' predicates. For instance, as discussed in the previous section, the content of the complement of the `semi-factive' predicate {\em discover} was significantly less projective than the content of the complement of the `factive' predicate {\em annoyed}. The predicate {\em know} is often considered a `factive' predicate but has also been suggested to show some parallels with `semi-factive' predicates (see, e.g., \citealt{kiparsky-kiparsky71,levinson83,simons01,beaver-geurts-sep}). Even though in Exp.~1a the content of the complement of {\em know} was more projective than the content of the complement of {\em discover}, and statistically indistinguishable in projectivity from the content of the complement of {\em annoyed}, the mean projectivity was numerically lower: .92 {\em know} vs.\ .96 {\em annoyed}. While these findings align with reported intuitions about differences between `factive' and `semi-factive' predicates, other findings of Exp.~1b do not reflect this division between predicates. For instance, the projectivity of the contents of the complements of the emotive `factive' predicates {\em be annoyed} and {\em be amused} was indistinguishable in Exp.~1b from that of the contents of the complements of the `semi-factive' predicates {\em notice, be aware, realize, see, find out} and {\em learned}. It is possible, of course, for a different diagnostic for projection to bring out further distinctions among the projective contents of these predicates.

Whereas the contents of the complements of the aforementioned 8 predicates were generally robustly projective, and that of {\em discover} slightly less projective, it was the contents of the complements of the cognitive `semi-factive' predicate {\em reveal} and of the communication `semi-factives' {\em confess} and {\em establish} that were significantly less projective than that of the other 9 predicates. Thus, whereas the contents of the complements of some of the predicates classically considered to be `semi-factive' were quite robustly projective in Exp.~1b, the contents of the complements of these three comparatively less discussed predicates emerged as clearly less projective. The traditional binary division of predicates into `factive and `semi-factive' is not straightforwardly mapped onto the observed differences among predicates in our experiments.

The previous literature is not always in agreement about the status of the contents of the complements of {\em reveal, confess} and {\em establish}. For {\em establish}, \citet{wyse} took the content of its complement to be projective, but \citet{swanson2012} classified the content as entailed, but not projective -- interestingly, he proposed the same for {\em discover}. The results of Exp.~1b suggest that the content of the complement of {\em establish} should be considered projective content, compared to, for instance, non-projective main clauses -- and likewise for {\em discover}. For {\em reveal}, the content of the complement was taken to be projective by \citet{hooper1974} and \citet{melvold1991}, and the findings of Exp.~1b support this assumption. Finally, for {\em confess}, authors generally take the content of its complement to be projective (e.g., \citealt{reis1973,melvold1991,schultz2003,swanson2012,karttunen2016}; cf.\ \citealt{wyse}), but \citet{swanson2012} suggested that the content of the complement is not entailed, giving a variant of the example in (\ref{confess}):\footnote{In Swanson's original example, {\em confess} did not combine with a finite complement introduced by {\em that}, but with an infinitival complement introduced by {\em to}: {\em She confessed to taking the money\ldots.}}

\begin{exe}
\ex\label{confess} She confessed that she took the money, but later recanted. It turned out that she had been trying to cover up a friend's mistake. \hfill (adapted from \citealt[\jt{pages}]{swanson2012})
\end{exe}
%The idea that the content of the clausal complement of {\em confess} is not entailed is also supported by the acceptability of sentences with {\em falsely confessed} (as readily shown by a Google search), like (\ref{falsely}):
%
%\begin{exe}
%\ex\label{falsely} At that point, he said, he falsely confessed that he had taken \$10 from an undercover officer in a marijuana sale.\footnote{\url{http://www.nytimes.com/1985/04/22/nyregion/youth-s-charges-of-torture-by-an-officer-spur-inquiry.html}}
%\end{exe}
%However, the attitude holder in (\ref{confess}) recants the content of the complement of {\em confess}, i.e., she revises her beliefs about the content. As such, (\ref{confess}) does not seem the best kind of example for identifying whether the content of the complement of {\em confess} is entailed. Clearer examples that show that the content of the complement of {\em confess} is not entailed are not difficult to find. Consider, for instance, the naturally occurring example in (\ref{confess2}):
%
%\begin{exe}
%\ex\label{confess2} Bagley initially confessed that he took the goods, then later denied the thefts, blaming them on a “shady” acquaintance.\footnote{\url{http://nypost.com/2013/04/08/pretty-little-liars-actor-pleads-guilty-to-misdemeanor-petit-larceny/}}
%\end{exe}
Swanson's idea that the content of the complement of an attitude predicate can be project even if it is not entailed content is also found in \citet{best-question} for the content of the complement of {\em believe}. Our findings for {\em confess} certainly provide initial support for this idea though a thorough investigation of the extent to which non-entailed clausal complements can be projective is a topic for future research.

Finally, recall that \citet{xue-onea11} found that the content of the complement of the German predicate {\em wissen} `know' was less projective than that of {\em erfahren} `find out'. In our experiments, by contrast, the mean projectivity of the contents of the complements of {\em discover} and {\em find out} were lower ({\em discover} .86 (Exp.~1a) and .85 (Exp.~1b), {\em find out} .88, Exp.~1b) than the projectivity of the content of the complement of {\em know} (.92, Exp.~1a). While these findings may be suggestive of cross-linguistic variation, it is important to note that there are several differences between \citetpos{xue-onea11} experiment and ours: in their study, the predicates were embedded in the antecedents of conditionals rather than in polar questions; participants were asked to judge whether it is possible that the content of the complement is false rather than whether the speaker is certain of the content of the content of the complement; and, finally, the projective contents were instantiated with different lexical contents. Exploring potential cross-linguistic variation in projectivity is an exciting area for future research.

Regarding the second research question, whether at-issueness plays a role in projectivity, Exps.~1a and 1b both provided empirical support for such a role, as predicted by  \citepos{brst-ar} Projection Principle. Specifically, we found that the information-structural status of the 19 projective contents tested was related to the projectivity of these contents. This finding significantly substantiates the ``clear correlation between projection and not-at-issueness'' that was suggested in \citealt[180]{xue-onea11} based on two pilot experiments involving four German expressions associated with projective content. Furthermore, Exps.~1a and 1b both suggest that the expression that the projective content is associated with plays a role in the projectivity of such content, and that speakers of American English differ in the extent to which they judge projective content to project. Exp.~1a also suggests that the lexical content that instantiates the projective content contributes to the projectivity of such content. Thus, in sum, the findings of Exps.~1a and 1b suggest that the projectivity of utterance content is a function of several factors, including the information structural status of such content, the lexical content that instantiates such content, the expression associated with the projective content and the individual interpreting the utterance. We discuss the implications of these findings for analyses of projection in section \ref{s5}, after further testing the Projection Principle in Exps.~2.

Our exploration of the Projection Principle in Exps.~1 relied on exploring the at-issueness of projecting content using the `asking whether' diagnostic. This diagnostic was chosen to assess at-issueness because i) it was used in previous research on at-issueness (e.g., \citealt{amaral-etal07,tonhauser-sula6}) and ii) the diagnostic is suitable to diagnose the at-issueness of projective content associated with expressions realized in polar questions (recall that the expressions were embedded in polar questions to assess projectivity). However, a potential worry is that the `asking whether' at-issueness diagnostic and the `certain that' projectivity diagnostic seem to mirror each other. After all, if Patrick, after uttering the polar question in (\ref{stim}), is taken to be certain that Martha's new car is a BMW, then he is presumably not asking whether her new car is a new BMW, and if he is taken to be asking whether Martha's new car is a BMW, then he is presumably not certain that her new car is a BMW.

\begin{exe}

\exi{(\ref{stim})} Patrick asks: {\em Was Martha's new car, a BMW, expensive?} 

\begin{xlist}
\ex `certain that' question: Is Patrick certain that Martha's new car is a BMW?

\ex `asking whether' question: Is Patrick asking whether Martha's new car is a BMW?

\end{xlist}

\end{exe}
Luckily, several different diagnostics have been used in the literature to diagnose at-issueness (see, e.g., \citealt{tonhauser-sula6}; for further discussion see section \ref{s-disc2}). The second pair of experiments, Exps.~2a and 2b, explored the at-issueness of the 19 projective contents using a different diagnostic for at-issueness to establish that the empirical support for the Projection Principle obtained in Exps.~1 is not merely an artifact of the at-issueness diagnostic used.

\section{Confirming the role of information structure in projectivity}\label{s4}

In Exp.~2, the at-issueness of the 19 projective contents was explored with a diagnostic that relies on the assumption that at-issue and not-at-issue content differ in the extent to which it is up for debate and can be directly assented/dissented with. For previous uses of diagnostics that rely on this assumption see, e.g., \citealt{amaral-etal07,xue-onea11,murray2014,anderbois-etal2015,destruel-etal2015,tonhauser-sula6} and \citealt{syrett-koev2015}. The 3-turn dialogue in (\ref{sure}) illustrates how the diagnostic was set up, on the basis of the appositive content of nominal appositives. The speaker of the first turn, here Fred, utters an indicative sentence with the target expression, here a nominal appositive, and thereby commits himself to various utterance contents, including the projective content. The speaker of the second turn, here Carla, utters the question {\em Are you sure?}~and thereby challenges some content of the first speaker's utterance. In the third turn, the speaker of the first turn utters an indicative sentence in which the content to be diagnosed for at-issueness, here the appositive content of the first turn, realizes the content of the clausal complement of {\em sure}, thereby identifying it as the content that they took the second speaker to be challenging. 

\begin{exe}
\ex\label{sure} 
\begin{xlist}
\exi{Fred:} Martha’s new car, a BMW, was expensive.

\exi{Carla:} Are you sure?

\exi{Fred:} Yes, I am sure that Martha's new car is a BMW.
\end{xlist}
\end{exe}
To assess whether participants take the 19 projective contents to be up for debate and a possible target of the second speaker's dissent, we asked them to respond to the question of whether the first speaker answered the question of the second speaker, using a slider from `no' to `yes'. In (\ref{sure}), for instance, the question participants responded to was whether Fred answered Carla's question. A `no' response was taken to indicate that the participant took Carla to have challenged a different content and Fred to therefore not have answered Carla's question; in this case, the projective content was not at-issue. A `yes' response, on the other hand, was taken to indicate that the participant took Carla to have challenged the projective content and Fred to have answered Carla's question, and, therefore, that the projective content was at-issue.\footnote{As mentioned above, our second at-issueness diagnostic relies on the assumption that at-issue and not-at-issue content differ in the extent to which it is up for debate and can be directly assented/dissented with. Our diagnostic differs from diagnostics used in prior research based on the same assumption because we wanted to explore the at-issueness of a broader set of projective contents and, in particular, ones that are not independent of the main clause content.  \citet{xue-onea11}, for instance, presented participants with indicative sentences with the target expressions, as did we, but asked participants to choose between (German versions of) `Yes, and \ldots' followed by a clause that denies the projective content and `No, (but) \ldots' followed by a clause that denies the projective content.  \citet{syrett-koev2015} also presented participants with indicative sentences with the target expressions, but asked participants to choose between a direct dissent utterance `No, \ldots' followed by a clause that denies the projective content and a direct dissent utterance `No, \ldots' followed by a clause that denies the main clause content. These diagnostics are less suitable for, e.g., indicative sentences with {\em know}, like {\em Billy knows that Martha has a new BMW}, because it is not possible to agree with the truth of the main clause content and simultaneously deny the truth of the content of the complement clause (one of the response options in \citepos{xue-onea11} diagnostic) and because denying the truth of the content of the complement also denies the truth of the main clause content (one of the response options in \citepos{syrett-koev2015} diagnostic).}

The at-issueness diagnostic used in Exp.~2 thus differs from the one used in Exp.~1 in several ways: i) the target expressions are realized in indicative sentences rather than in polar questions, i.e., not embedded under an entailment-canceling operator; ii) the diagnostic relies on the assumption that at-issue and not-at-issue content differ in the extent to which it is up for debate and can be dissented with, rather than in the extent to which it and its negation partition the context set; and iii) participants were asked to rate the extent to which an utterance answered a question, rather than what a speaker is asking about. 

Because the target expressions in Exp.~2 are not realized in the scope of an entailment-canceling operator, we cannot collect projectivity ratings for the projective contents associated with the target expressions. To assess whether the at-issueness of the 19 projective contents under this second diagnostic plays a role in their projectivity, we collected at-issueness ratings for the same combinations of target expressions and lexical contents as in Exp.~1, and then related the mean at-issueness rating of each combination to the mean projectivity rating of the same combination from Exp.~1. If projectivity is a function of at-issueness, we expect this second diagnostic to further confirm the relation between projectivity and at-issueness observed in Exp.~1. We present the results of Exps.~2a and 2b in sections \ref{s-exp2a} and \ref{s-exp2b}, and then discuss the results and compare the two at-issueness diagnostics in section \ref{s-disc2}.

\subsection{Experiment 2a}\label{s-exp2a}

Exp.~2a explored the at-issueness of the 9 projective contents that we explored in Exp.~1a, i.e., the contents of NRRCs and nominal appositives, the possession implication of possessive noun phrases, the prejacents of {\em only} and {\em stupid}, the pre-state implication of {\em stop} and the contents of the clausal complements of {\em annoyed, discover} and {\em know}, using the {\em Are you sure?}~diagnostic introduced above.

\subsubsection{Methods}\label{s-methods-2a}

\paragraph{Participants.} 250 participants with U.S.\ IP addresses and at least 97\% of previous HITs approved were recruited on Amazon's Mechanical Turk platform (ages: 20-77; median: 30). They were paid 30 cents for their participation.


\paragraph{Materials.} Stimuli consisted of written 3-turn dialogues between two individuals, as in (\ref{sure}). In the target stimuli, the first turn of each dialogue consisted of an indicative sentence that realized one of the 9 target expressions. The projective contents of these 9 target expressions were instantiated by the same 17 lexical contents as in Exp.~1a (see section \ref{s-methods-1a}). Thus, there were a total of 43 indicative sentences with target expressions that realized the first turn of the target stimuli. The second turn of the target stimuli consisted of a second speaker's {\em Are you sure?}~question and the third turn consisted of an utterance by the first speaker in which {\em Yes, I am sure that} was followed by a clause that realized the projective content to be diagnosed.

As in Exp.~1a, there were 17 control stimuli in Exp.~2a: in the control stimuli, the first turn consisted of an indicative sentence that realized one of the 17 lexical contents and, in the third turn, the clause that realized the lexical content was the complement of {\em sure}. A sample control dialogue is shown in (\ref{sure2}). The names of the two speakers in the dialogues were randomly selected. The full set of stimuli of Exp.~2a is provided in Appendix \ref{a-exp1a-2a-stimuli}.


\begin{exe}
\ex\label{sure2}
\begin{xlist}
\exi{Sandra:} Martha has a new BMW.

\exi{Carl:} Are you sure?

\exi{Sandra:} Yes, I am sure that Martha has a new BMW.
\end{xlist}
\end{exe}

For each participant, a set of 15 stimuli was randomly created: each set contained a target stimulus for each of the 9 target expressions (the projective content of each expression was instantiated by a unique lexical content) and 6 control stimuli (with unique lexical contents as well, for a total of 15 unique lexical contents from (\ref{contents})). Trial order was randomized for each participant.


\paragraph{Procedure.} Participants were told to imagine that they are at a party and, upon walking into the kitchen, overhear a short conversation between two people. Participants were then presented with the 15 stimuli in random order and were asked to assess, for each stimulus, whether the speaker of the first/third turn answered the question of the speaker of the second turn. Participants gave their responses on a slider marked `no' at one end and `yes' at the other, as shown in Figure \ref{f-trial-exp2a}. A `yes' response was taken to indicate that the relevant content was at-issue and a `no' response that the relevant content was not at-issue. To explore the hypothesis that projectivity and not-at-issueness are positively related, `yes' responses were coded as `0' and `no' responses as `1'. 

\begin{figure}[!h]
\begin{center}
\fbox{\includegraphics[width=12cm]{figures/exp2-trial}}
\end{center}
\caption{A sample trial in Exp.~2a}
\label{f-trial-exp2a}
\end{figure}

After completing the experiment, participants filled out the same optional survey as in Exps.~1 about their age, their native language(s) and, if English is their native language, whether they are a speaker of American English (as opposed to, e.g., Australian or Indian English). To encourage them to respond truthfully, participants were told that they would be paid no matter what answers they gave in the survey.

\paragraph{Data exclusion.} Prior to analysis, we excluded the data from 6 participants who did not self-identify as native speakers of American English. Inspection of the response means of the remaining 244 American-English speaking participants to the 6 control stimuli revealed 6 participants whose response means were more than 3 standard deviations above the group mean (which was 0.04). Further inspection revealed that these participants' responses were systematically higher than the group mean and involved 14 of the 17 lexical contents, suggesting that these participants did not attend to the task or interpreted the task differently. The data from these 6 participants were also excluded, leaving data from 238 participants (ages 20-77; median: 30).


\subsubsection{Results}

Mean projectivity ratings obtained for target expression/lexical content combinations in Exp.~1a are shown in \figref{fig:f-proj-ai-2a} as a function of their mean not-at-issueness ratings obtained in Exp.~2a. There is a clear relationship between at-issueness and projectivity: the more not-at-issue a projective content is, as measured by the second at-issueness diagnostic, the more projective it is.

\begin{figure}[!h]

\begin{center}
%\includegraphics[width=12cm]{../results/exp2a/graphs/ai-proj-bytrigger}
\includegraphics[width=10cm]{../results/exp2a/graphs/ai-proj-bytrigger-labels}
\end{center}

\caption{Mean projectivity against mean not-at-issueness by target expression. Error bars indicate bootstrapped 95\% confidence intervals. Dashed line indicates perfect correlation line.\jt{Include `sure that' diagnostic in x-axis label, as above?}}
\label{fig:f-proj-ai-2a}
\end{figure}


The observed relationship was borne out statistically. We conducted a similar mixed-effects linear regression analysis as we did for Exp.~1a, predicting projectivity from a centered fixed effect of at-issueness and random by-lexical content intercepts. This model differed from that in Exp.~1a in the following three ways: i) because we predicted projectivity ratings given by one group of participants (Exp.~1a) from at-issueness ratings given by another group of participants (Exp.~2a), by-participant random effects were not included. Instead, the model predicted projectivity \emph{means}  from  at-issueness \emph{means}, \jt{collapsing across participants but not lexical contents}; ii) there was no fixed effect of block because block was not manipulated; iii) by-expression random effects and random by-lexical content slopes for at-issueness were not included because likelihood ratio tests revealed that the only random effect justified by the data was that of by-lexical content intercepts ($SD$ = .04, $p < $ .05). The model we report here thus only contained one fixed effect (at-issueness) and one random effect (by-lexical content intercepts).

We observed a significant main effect of at-issueness, such that more not-at-issue expression/lexical content combinations received higher projectivity ratings ($\beta$ = 0.29, $SE$ = 0.06, $t$ = 5.21, $\chi^2(1)$ = 20.94, $p <$ .0001), replicating the at-issueness effect observed in the previous experiments.\footnote{In the model with the full random effects structure (random by-lexical content and by-expression intercepts and slopes for at-issueness), the effect of at-issueness was only marginally significant ($\beta$ = 0.25, $SE$ = 0.08, $t$ = 3.15, $\chi^2(1)$ = 3.51, $p <$ .07).  A post hoc power analysis using the simr package \citep{simr} revealed that this model had only 60.4\% power to detect an effect size of $\beta$ = .25, while the model with the simplified random effects structure had 99.8\% power to detect an effect size of $\beta$ = .29. This means that the dataset is not large enough to jointly estimate the effects of at-issueness and the random effects. However, the fact that at-issueness is a marginally significant predictor of projectivity even in this low-powered dataset provides further evidence for the relation between at-issueness and projectivity.}

\subsection{Experiment 2b}\label{s-exp2b}

Exp.~2b explored the at-issueness of the 12 projective contents that we explored in Exp.~1b, namely the contents of the clausal complements the predicates {\em be amused, be annoyed, know, be aware, see, discover, find out, realize, learn, establish, confess} and {\em reveal}, using the {\em Are you sure?}~diagnostic.

\subsubsection{Methods}

\paragraph{Participants.} 250 participants with U.S.\ IP addresses and at least 97\% of previous HITs approved were recruited on Amazon's Mechanical Turk platform (ages: 18-77; median: 29). They were paid 30 cents for their participation.

\paragraph{Materials.} As in Exp.~2a, the stimuli consisted of 3-turn dialogues between two individuals. In the target stimuli, the first turn of each dialogue consisted of an indicative sentence that realized one of the 12 predicates, as shown in (\ref{sure3}). The contents of the complements of these predicates were instantiated by the same 20 lexical contents as in Exp.~1b (see section \ref{s-methods-2a}), for a total of 240 target stimuli. The third turn of the target stimuli consisted of the first speaker's utterance of {\em Yes, I am sure that}, with the relevant projective content realized as the content of the complement of {\em sure}. 

\begin{exe}
\ex\label{sure3}
\begin{xlist}
\exi{Sandra:} Shirley is aware that Raul was drinking chamomile tea.

\exi{Carl:} Are you sure?

\exi{Sandra:} Yes, I am sure that Raul was drinking chamomile tea.
\end{xlist}
\end{exe}

As in Exp.~1b, there were 20 control stimuli in Exp.~2a: in the control stimuli, the first turn consisted of an indicative sentence that realized one of the 20 lexical contents and, in the third turn, the clause that realized the lexical content was the complement of {\em sure}. A sample control stimulus is shown in (\ref{sure4}).

\begin{exe}
\ex\label{sure4}
\begin{xlist}
\exi{Sandra:} Raul was drinking chamomile tea.

\exi{Carl:} Are you sure?

\exi{Sandra:} Yes, I am sure that Raul was drinking chamomile tea.
\end{xlist}
\end{exe}

For each participant, a set of 20 stimuli was randomly created: each set contained a target stimulus for each of the 12 target expressions (the projective content of each expression was instantiated by a unique lexical content) and 8 control stimuli (with unique lexical contents as well, for a total of 20 unique lexical contents from (\ref{contents2})). Trial order was randomized for each participant.

\paragraph{Procedure.} The procedure was the same as in Exp.~2a, described in section \ref{s-methods-2a}, except that participants completed 20 trials instead of 15.

\paragraph{Data exclusion.} Prior to analysis, we excluded the data from 6 participants who did not self-identify as native speakers of American English. Inspection of the response means of the remaining 244 American-English speaking participants to the 8 control stimuli revealed 6 participants whose response means were more than 3 standard deviations above the group mean (which was 0.05). Further inspection revealed that these participants' responses were systematically higher than the group mean and involved 18 of the 20 lexical contents, suggesting that these participants did not attend to the task or interpreted the task differently. The data from these 6 participants were also excluded, leaving data from 238 participants (ages 18-77; median: 30).


\subsubsection{Results}

Mean projectivity ratings obtained for target expression/lexical content combinations in Exp.~1b are shown in \figref{fig:f-proj-ai-2b} as a function of their mean not-at-issueness ratings obtained in Exp.~2b. While there appears to be an overall increase in projectivity with increasing not-at-issueness, as predicted by the Projection Principle, the relationship is not clearly linear, unlike in the previous experiments. 

\begin{figure}[!h]

\begin{center}
%\includegraphics[width=12cm]{../results/exp2b/graphs/ai-proj-bytrigger}
\includegraphics[width=10cm]{../results/exp2b/graphs/ai-proj-bytrigger-labels}
\end{center}

\caption{Mean projectivity against mean not-at-issueness by target expression. Error bars indicate bootstrapped 95\% confidence intervals. Dashed line indicates perfect correlation line. \jt{Include that `are you sure' diagnostic on x-axis label?}}
\label{fig:f-proj-ai-2b}
\end{figure}


%Formula: mean_proj ~ cmean_ai + (1 | short_trigger)
%   Data: means_nomc
%
%     AIC      BIC   logLik deviance df.resid 
%  -554.1   -540.1    281.0   -562.1      236 
%
%Scaled residuals: 
%    Min      1Q  Median      3Q     Max 
%-3.0835 -0.5640  0.1461  0.6739  2.4186 
%
%Random effects:
% Groups        Name        Variance Std.Dev.
% short_trigger (Intercept) 0.020061 0.14164 
% Residual                  0.004494 0.06704 
%Number of obs: 240, groups:  short_trigger, 12
%
%Fixed effects:
%            Estimate Std. Error t value
%(Intercept)  0.83092    0.04112  20.209
%cmean_ai     0.03320    0.04014   0.827
%
%Correlation of Fixed Effects:
%         (Intr)
%cmean_ai 0.000 

A mixed effects linear regression predicting mean projectivity from mean at-issueness and random by-expression intercepts (the only random effect term justified in likelihood ratio tests) \jt{reveals a small albeit positive coefficient} but yielded no significant effect of at-issueness ($\beta$ = 0.03, $SE$ = 0.04, $t$ = 0.83, $\chi^2(1)$ = 0.68, $p >$ .4). 

\jt{Need to include information on significance for by-expression intercept in text here, and then also in summary table in Discussion section}

\jt{Power analysis?}


\newpage

bla

\newpage

\subsection{Summary and discussion of Experiments 2a and 2b}\label{s-disc2}

Exps.~2a and 2b were designed to further test the Projection Principle (research question \ref{questions}b) by exploring the at-issueness of the 19 projective contents explored in this paper using a second diagnostic for at-issueness. We found that at-issueness was a significant predictor of the projectivity of the 9 projective contents in Exp.~2a but not of the projectivity of the 12 projective contents in Exp.~2b. Thus, only Exp.~2a provided further empirical evidence for the Projection Principle.

\jt{Question: Is it noteworthy that lexical content mattered in Exp 2a but not in Exp 2b, parallel to Exp 1a versus 1b, or am I right that we did not replicate the pattern here but only re-established it using on the basis of means rather than individual participants' responses?}

%The two experiments also differed in which factors other than at-issueness were found to play a role in projectivity. In Exp.~2a, the lexical contents with which the 9 projective contents were instantiated was found to play a role in the projectivity of the projective contents, but not the lexical contents that instantiated the 20 projective contents in Exp.~2b. This pattern was already found in Exp.~1 where the lexical contents significantly predicted projectivity in Exp.~1a but not in Exp.~1b.
%
%
%This difference between the experiments may be due to a number of factors. First, different lexical contents were included in the two experiments and it is possible that the lexical contents in Exp.~1a were more heterogeneous than those in Exp.~1b. Second, the lexical contents uniformly instantiated the contents of clausal complements in Exp.~1b, but a quite varied set of projective contents in Exp.~1a, which may contribute to the variability observed between the lexical contents. Third, recall that in Exp.~1b, each lexical content was paired with every target expression, whereas in Exp.~1a, each lexical content was paired with only a subset of target expressions. As a result, there were 3-21 data points per lexical content/expression combination in Exp.~1b, but 12-78 data points per lexical content/expression combination in Exp.~1a. This difference in number of ratings obtained may have contributed to the observed difference between the two experiments in by-lexical content variability. The role of the lexical contents in projectivity merits further investigation, which we leave to future research. 

\jt{Revise this:  In Exp.~2b, by contrast, the lexical contents were not found to play a role. These findings parallel what was already observed in Exps.~1a and 1b, respectively. As discussed in section \ref{s-summary1a1b}, we leave the exploration of this difference between Exps.~1a/2a and Exps.~1b/2b for future research.
Exps.~2a and 2b also differed in whether the target expression was a significant predictor of the projectivity of the projective content: target expression was not a significant predictor in Exp.~2a but in Exp.~2b. This finding from Exp.~2b further confirms what Exps.~1 already suggested, namely that the conventional contents of the target expressions that are associated with projective content play a role in the projectivity of the projective content. We discuss the implications of this finding for empirically adequate analyses of projection in section \ref{s5}.}


\paragraph{Comparison of the at-issueness diagnostics used in Exps.~1 and 2.} In the remainder of this section we compare the two at-diagnostics measures used in Exps.~1 and 2.  Mean at-issueness ratings obtained in Exp.~1 are shown in \figref{fig:ai-correlation} as a function of their mean at-issueness ratings obtained in Exp.~2. There is a clear relationship between the two at-issueness diagnostics: the more not-at-issue a projective content is on the {\em Are you sure?}~diagnostic used in Exp.~2, the more not-at-issue it is on the `asking whether' diagnostic used in Exp.~1.

\begin{figure}[!h]
\begin{center}

\includegraphics[width=14cm]{../results/ai-meta-analysis/graphs/correlation-bytrigger}

\end{center}
\caption{Mean not-at-issueness ratings for each target expression in Exps.~1 and Exps.~2. Error bars indicate bootstrapped 95\% confidence intervals. Dashed line indicates perfect correlation line.}
\label{fig:ai-correlation}
\end{figure}


\begin{itemize}

\item This observation was borne out statistically. \jt{JD, what's the best to do here? Overall correlation at the trigger level (collapsing across contents): .62. Not collapsing across contents: .31. Correlations separately by sub-experiment at the trigger level: a) syntactically heterogenous .70; b) syntactically homogeneous .56. Not collapsng across contents:  a) syntactically heterogenous .53; b) syntactically homogeneous .27}

\item Differences between the diagnostics:

\begin{itemize}

\item \jd{discuss shallower slope, greater variability in ai measure}

\item e.g., {\em discover} versus {\em stop} (others?)

\item generally higher at-issueness in `are you sure?' diagnostic than in `asking whether' diagnostic suggests that projective content may be more available as antecedent to anaphor than be able to partition the context set


\end{itemize}

\item Comparison with other experimental results:

\begin{itemize}

\item \citealt{xue-onea11}: German wissen is more at-issue than find out, used direct dissent diagnostic

\item \citealt{amaral-etal11}, \citealt{cummins-etal2012}: British English, prejacent of {\em only} is more at-issue than the post- and pre-state implications of {\em continue} and {\em stop}, respectively.

triggers: again, stop, continue, only, comparative construction

triggers embedded under polar questions

task: how natural is `no, p' response that denies foregrounded/AI versus projective/NAI content?

\item In our experiments, too, there were differences in at-issueness:  

Exp 1a: stop $<$ only $<$ discover $<$ know $<$ annoyed \jt{Do stats, or how discuss?}

Exp 2a: only $<$ discover $<$ know $<$ stop $<$ annoyed

\item \citealt{syrett-koev2015}: used dissent diagnostic, found that NRRCs and nominal appositives can be at-issue, extent to which they are at-issue depends on position.

We only did sentence-medial position. Exp 1a: NRRCs and nominal appositives are pretty much at ceiling, little by-participant variability; Exp 2a: NRRCs and nominal appositives still among the 4 most not-at-issue projective contents, but because overall higher at-issueness in Exp2, mean rating for NRRCs and nominal appositives is now around .8. 

\end{itemize}

\item Some remarks on how at-issueness has not been defined and hence we don't know which diagnostic gets at `real' at-issueness.

At-issue content was characterized in \citealt{brst-salt10} as utterance content that is (at least) addresses, i.e., contextually entails an answer to, the Question Under Discussion (QUD, \citealt{roberts12}) addressed by the utterance, but other authors have given other characterizations of at-issue content: in \citealt{potts05}, at-issue entailments are ``regular asserted content (`what is said' in Grice's terms)'' (p.6)\jt{Check page number}

\begin{itemize}

\item At least 5 different assumptions are made about at-issueness in at-issueness diagnostics that are currently being used in the literature.

\item We have used two diagnostics here that rely on distinct assumptions about at-issueness (partition, anaphoricity). Both revealed an influence of at-issueness on projectivity.

\item Clearly an important topic for future research: i) what's a formal characterization of at-issueness? ii) which properties of at-issue and not-at-issue content derive from this formal characterization? iii) how can these properties be diagnosed with theoretically untrained speakers? 

\end{itemize}

\end{itemize}

\newpage

\begin{figure}[!h]
\begin{center}

%\includegraphics[width=16cm]{../results/exp2a/graphs/ai-subjectmeans}
%\includegraphics[width=16cm]{../results/exp2b/graphs/ai-subjectmeans}

\includegraphics[width=10cm]{../results/exp1a/graphs/boxplot-not-at-issueness}
\includegraphics[width=16cm]{../results/exp1b/graphs/boxplot-not-at-issueness}

\includegraphics[width=10cm]{../results/exp2a/graphs/boxplot-not-at-issueness}
\includegraphics[width=16cm]{../results/exp2b/graphs/boxplot-not-at-issueness}

\end{center}
\caption{At-issueness by participant (top panel) and projective content trigger (bottom panel)}
\label{f-ai-2b}
\end{figure}

\begin{figure}[!h]
\centering

%\subfloat[][Boxplot of at-issueness variability by target expression, collapsing across  lexical contents. Blue dots indicate means and notches indicate medians.]{ 
%\includegraphics[width=12cm]{../results/exp2a/graphs/boxplot-not-at-issueness}
%\label{fig:proj-aimeans-2a}
%}

%\subfloat[][At-issueness means by participant. Error bars indicate bootstrapped 95\% confidence intervals.]{ 
%\includegraphics[width=12cm]{../results/exp2a/graphs/ai-subjectmeans}
%\label{fig:ai-subjmeans-2a}
%}


\caption{At-issueness by target expression (top panel) and by participant (bottom panel)\jd{OK, I think I changed my mind and we should be showing these at-issueness plots for Exps 1 as well. But maybe next to their matched plots here, for comparison? I continue to think they're not so interesting to show in the Exp 1 sections, but as a point of comparison here it might be helpful?}}
\label{fig:f-ai-2a}
\end{figure}


\newpage

\section{Discussion}\label{s5}

a. there’s projection variability, comparison with previous experiments and researchers’ intuitions

there’s evidence from 3 experiments for at-issueness playing a role in projection

there’s also evidence that target expressions and lexical contents play a role, and participant variability

implications for analyses of projection

\begin{exe}
\exi{(\ref{questions})} {\bf Research questions}

\begin{xlist} 

\ex Does projective content vary in how robustly it projects?

\ex Is projectivity a function of at-issueness, as predicted by the Projection Principle?
\end{xlist}

\end{exe} 

cite CogSci paper and SALT paper to discuss influence of prosody, also \citealt{best-question}

the fact that the meaning of the trigger matters is not immediately support for conventional triggering analyses: it could also be that overarching properties of the triggers matter (e.g., emotive versus cognitive versus verb of saying).

projectivity may be due to different reasons for different triggers: conventional specification for SCF+, sensitivity to QUD for only and manner adverb utterances, etc.

\paragraph{Projection variability.}

Karttunen: factive versus non-factive predicates

Abusch: hard versus soft triggers

Xue \& Onea: wissen is less projective than find out

Smith \& Hall: win and know are less projective than content implication of definite NP


\paragraph{Projectivity as a function of at-issuess}


\begin{table}[h!]

\begin{center}
\begin{tabular}{r | c | c c c | c c c c}
\toprule
& & \multicolumn{3}{c|}{Intercepts} & \multicolumn{3}{c}{Slopes for at-issueness}\\
& At-issueness & Expression & Content & Participant & Expression & Content & Participant\\
\midrule
Exp.~1a & ** & *** & ** & *** & *** & *** & ***  \\ 

Exp.~1b & *** & *** & \emph{n.s.} &  *** & \emph{n.s.} & \emph{n.s.} & *** \\ 

Exp.~2a & *** & \emph{n.s.} & * & -- & \emph{n.s.} & \emph{n.s.} & -- \\ 

Exp.~2b & \emph{n.s.} & \jt{missing} & \emph{n.s.} & --& \emph{n.s.} & \emph{n.s.} & -- \\ 
\bottomrule
\end{tabular}
\label{t-summary}
\end{center}
\caption{Summary of findings in Exps.~1 and 2. `Expression' abbreviates target expression, `Content' abbreviates lexical content. `***' indicates significance at .0001, `**' at .01, `*' at .05, `.' marginal significance at .1, \emph{n.s} indicates no significance and `--' indicates `not applicable'.}
\end{table}

Exp 2b: coefficient goes into the right direction for at-issuenesss

Power problems in Experiments 2a and 2b

\paragraph{Implications for analyses of projection.}

options: 1. projection is conventionally specified, at-issueness influences this to predict variability (keeping with conventionalist approaches), 2. projection is not conventionally specified but determined by at-issueness (simons et al 2010), 3. a mix of 1 and 2

\begin{itemize}

\item Can theories of projection predict our findings (variability, relation to AIness)? We think our findings show that Projection Principle is true, and now discusses how that might come about.

\item This paper doesn't test whether triggering is conventional or not, though it is suggestive; we have the following hypothesis: conventional non-variability hypothesis says that for a conventional trigger you won't get variation in projectivity (implicit); under this hypothesis (which we don't test), the results do indeed mean that some triggers are conventional and others are not; make explicit under which conditions we can make which inferences


\item Kartunen plugs holes filters: doesn't explain any cases where you don't get projection from a projection (variability, AI), embedding environment is crucial

\item Gazdar: you'll get projection unless you have a contradiction; no contradictions in our contexts; so again no explanation for our data; theory predicts robust projection

\item heim/vds: you'll get projection unless either there's a contradiction or a lack of informativeness; again, neither applies in our data, so they can't explain the data

\item Abusch:

\item Abrusan: 

\item Partee: information structure influences local accommodation (allegation and local accommodation)

\item Best question/SALT paper: suggest a hypothesis (extended projectivity hypothesis in AR paper), strong hypothesis that ainess inversely correlates with projectivity; we think that there's a mix of conventional and conversational triggering; we argue in BQ paper that you could do with only conversational trigger, but we haven't settled the case yet whether there is conventional triggering

\item Occam's Razor: conventional + accommodation, or only conversational


\end{itemize}

Under conventionalist approaches to projection, the projectivity of content is derived from a conventionally specified requirement that the content be entailed by or satisfied in the common ground prior to utterance (e.g., \citealt{heim83,vds92}). Thus, for instance, conventionally requiring that the contents of the complements of {\em discover} and {\em annoyed} are entailed by the common ground of the interlocutors prior to utterance derives the observation that the contents of the complements project over entailment-canceling operators. However, such approaches to projection do not lead us to expect that the contents of the complements of these two predicates differ in how robustly they project since, after all, the two contents receive the same conventional specification. Furthermore, such approaches to projection do not lead us to expect that two speakers of American English will differ in how robustly they take the content of the complement of {\em discover} to project since, again, the content of the complement of {\em discover} is conventionally specified to project for both speakers. Thus, our finding that there is both between-item and between-participant variability in how robustly projective content projects is an empirical challenge to conventional approaches to projection. \jd{i don't think that we can sell the strongest anti-conventionalist story, though, given that not all of the variability can be reduced to at-issueness (which we know from the fact that the by-trigger random effects were justified) -- though it's possible at least in principle that the trigger effects can be reduced to the particular contents they occurred with, we also included by-lexical content random effects that ended up being justified in addition to the trigger random effects. That is, all these things are contributing independent variance.}

Conventionalist approaches are not helped here by appealing to local accommodation, the process by which projective content is accommodated in the scope of an entailment-canceling operator, e.g., to avoid contradictions, uninformativity or problems with binding (\citealt{heim83,vds92}). The context in which participants in Exp.~1a were asked to interpret the polar question stimuli was quite minimal since it only clarified that the participant overheard the speaker uttering the polar question upon entering the kitchen, at a party. Crucially, because the context was so minimal, the relevant projective contents cannot be argued to be locally accommodated to avoid contradictions with the context or to avoid uninformativity; and since the projective contents do not contain variables, problems with binding also do not warrant an appeal to local accommodation. Furthermore, in order for local accommodation to align conventionalist approaches to projection with the observed between-item and -participant variability, one would have to argue that, e.g., the content of the complement of {\em discover} is more likely to be locally accommodated than the content of the complement of {\em annoyed}, or that one participant is more likely to locally accommodate the content of the complement of {\em discover} than another participant. Absent a better understanding of local accommodation, such arguments are implausible. 

One way forward would be to develop a better understanding of local accommodation. Here we take a different route and suggest that the observed between-item and -participant variability provides empirical support for non-conventional approaches to projection, which attempt to derive projectivity from independently-motivated properties of projective content triggers and their use in context (e.g., \citealt{stalnaker74,kempson75,wilson75,boer-lycan76,levinson83,kadmon01,simons01,simons04,atlas05,abusch10,abrusan2011,best-question}). Under such approaches, between-item and -participant variability in projectivity could be attributed to differences between projective contents and their triggers. One such approach, as mentioned above, maintains that projective content differs in its information-structural properties, which has consequences for its projectivity: according to \citealt{brst-salt10} and \citealt{brst-ar}, projective content projects if and only if it is not at-issue. In Exp.~1a, the at-issueness of projective content was explored on the basis of a diagnostic for at-issueness that assumes that at-issue content differs from not at-issue content in how likely it (and its negation) are to partition the context set. The finding of Exp.~1a that the information-structural status of projective content, as diagnosed by this assumption about at-issueness, is a significant predictor of its projectivity provides empirical support the non-conventionalist approach to projection advanced in \citealt{brst-salt10} and \citealt{brst-ar}. In Exp.~1b, we extend our investigation of this hypothesis about projection to a wider range of projective content, namely the contents of the complements of (semi-)factive predicates.

\section{Conclusions}\label{s6}

\appendix

\section{Stimuli used in Experiments 1a and 2a}\label{a-exp1a-2a-stimuli}

The stimuli used in Exp.~1a are grouped here by the 17 lexical contents. For each content, the first line provides the label of the content (e.g., `muffins', for the first content). The second line (`Lexical content') identifies the lexical content. The remaining lines of each of the 17 lexical contents identify the projective content triggers that were instantiated by the content (e.g., `muffins' instantiated control stimuli, NRRCs and {\em only}). In Exp.~2a, indicative sentence variants of the polar questions were used.

\begin{enumerate}

\item  muffins:  \\
     Lexical content: these muffins have blueberries in them\\
     Control stimulus: Do these muffins have blueberries in them?\\
     NRRC: Are these muffins, which have blueberries in them, gluten-free and low-fat?\\
     {\em only}: Do these muffins only have blueberries in them?

\item pizza:  \\
     Lexical content: this pizza has mushrooms on it\\
     Control stimulus: Does this pizza have mushrooms on it?\\
     {\em only}: Does this pizza only have mushrooms on it?\\
     {\em annoyed}: Is Sam annoyed that this pizza has mushrooms on it?\\
     {\em discover}: Did Sam discover that this pizza has mushrooms on it?

\item play:  \\
     Lexical content: Jack was playing outside with the kids\\
     Control stimulus: Was Jack playing outside with the kids?\\
     {\em stop}: Did Jack stop playing outside with the kids?\\
     {\em know}: Does Daria know that Jack was playing outside with the kids?\\
     {\em discover}: Did Paula discover that Jack was playing outside with the kids?

\item veggie:  \\
     Lexical content: Don is a vegetarian\\
     Nominal appositive: Is Don, a vegetarian, going to find something to eat here?\\
     NRRC: Is Don, who is a vegetarian, going to find something to eat here?\\
     Control stimulus: Is Don a vegetarian?

\item cheat:  \\
     Lexical content: Raul cheated on his wife\\
     Control stimulus: Did Raul cheat on his wife?\\
     {\em know}: Does Daria know that Raul cheated on his wife?\\
     {\em stupid}: Was Raul stupid to cheat on his wife?

\item nails:  \\
     Lexical content: Mary's daughter has been biting her nails\\
     Control stimulus: Has Mary's daughter been biting her nails?\\
     {\em discover}: Did Mary discover that her daughter has been biting her nails?\\
     {\em stop}: Has Mary's daughter stopped biting her nails?\\
     {\em stupid}: Is Mary's daughter stupid to be biting her nails?

\item  ballet:  \\
     Lexical content: Ann used to dance ballet\\
     Control stimulus: Did Ann use to dance ballet?\\
     Nominal appositive: Is Ann, a former ballet dancer, limping?\\
     {\em stop}: Did Ann stop dancing ballet?

\item kids:  \\
     Lexical content: John's kids were in the garage\\
     {\em only}: Were John's kids only in the garage?\\
     Control stimulus: Were John's kids in the garage?\\
     {\em stupid}: Were John's kids stupid to be in the garage?

\item hat:  \\
     Lexical content: Samantha has a new hat\\
     Control stimulus: Does Samantha have a new hat?\\
     Possessive NP: Was Samantha's new hat expensive?\\
     {\em know}: Does Daria know that Samantha has a new hat?\\
     {\em annoyed}: Is Joyce annoyed that Samantha has a new hat?

\item bmw:  \\
     Lexical content: Martha has a new BMW\\
     Control stimulus: Does Martha have a new BMW?\\
     Possessive NP: Was Martha's new BMW expensive?\\
     Nominal appositive: Was Martha's new car, a BMW, expensive?\\
     {\em annoyed}: Is Martha's neighbor annoyed that Martha has a new BMW?\\
     {\em know}: Does Billy know that Martha has a new BMW?

\item boyfriend:  \\
     Lexical content: Betsy has a boyfriend\\
     Control stimulus: Does Betsy have a boyfriend?\\
     NRRC: Is Betsy, who has a boyfriend, flirting with the neighbor?\\
     Possessive NP: Is Betsy's boyfriend from around here?

\item alcatraz:  \\
     Lexical content: Mike visited Alcatraz\\
     Control stimulus: Did Mike visit Alcatraz?\\
     NRRC: Is Mike, who visited Alcatraz, a history fan?\\
     {\em discover}: Did Jane discover that Mike visited Alcatraz?\\
     {\em know}: Does Jane know that Mike visited Alcatraz?

\item aunt:  \\
     Lexical content: Janet has a sick aunt\\
     Control stimulus: Does Janet have a sick aunt?\\
     NRRC: Is Janet, who has a sick aunt, very compassionate?\\
     {\em know}: Does Melissa know that Janet has a sick aunt?\\
     Possessive NP: Has Janet's sick aunt been recovering?

\item cupcakes:  \\
     Lexical content: Marissa brought the cupcakes\\
     Control stimulus: Did Marissa bring the cupcakes?\\
     NRRC: Is Marissa, who brought the cupcakes, a good baker?\\
     {\em know}: Does Max know that Marissa brought the cupcakes?

\item soccer:  \\
     Lexical content: the soccer ball has a hole in it\\
     Control stimulus: Does the soccer ball have a hole in it?\\
     NRRC: Was the soccer ball, which has a hole in it, a gift from Uncle Bill?\\
     {\em annoyed}: Is Mandy annoyed that the soccer ball has a hole in it?\\
     {\em discover}: Did Mandy discover that the soccer ball has a hole in it?\\
     {\em know}: Does Mandy know that the soccer ball has a hole in it?

\item olives:  \\
   	Lexical content: this bread has olives in it\\
   	Control stimulus: Does this bread have olives in it?\\
   	{\em annoyed}: Is Barbara annoyed that this bread has olives in it?

\item stuntman:  \\
   	Lexical content: Richie is a stuntman\\
   	Control stimulus: Is Richie a stuntman?\\
   	Nominal appositive: Did Richie, a stuntman, break his leg?\\
   	{\em stupid}: Is Richie stupid to be a stuntman?

\end{enumerate}

\bibliographystyle{cslipubs-natbib}
\bibliography{bibliography}


\end{document}
